Le dernier intervenant est Alain Goergen, directeur des ressources humaines de la Police Fédérale. Sa carrière s'est construite au sein de la police. Il commence par suivre une formation d'officier de gendarmerie à l'école militaire avec une licence en criminologie. Après 2 ans sur le terrain il entame une seconde formation en management pour se réorienter vers un poste administratif. Il a été en charge de nombreux projets toujours au sein des ressources humaines comme le recrutement, la formation ou encore la restructuration complète du département RH.\\

La Police Fédérale est le plus grand employeur de Belgique. Elle compte 193 sections qui emploient plus de 50 000 personnes réparties dans différents domaines d'activités sur tout le territoire national. L'éventail de missions de la Police est assez large: maintien de l'ordre, assistance aux victimes, missions anti-terroristes, etc.\\
Malgré la division entre les 192 polices locales et la police fédérale, elles sont toutes dépendantes du département RH de la police fédérale.\\
 
La Police est aujourd'hui dans un phase difficile du point de vue économique et politique. En effet, le gouvernement doit faire des économies et ce service public n'est pas épargné. Le budget de la Police est donc moindre que les années précédentes %[\textbf{trouver chiffres?} http://www.sudinfo.be/1166936/article/2014-12-10/impact-de-la-baisse-de-budget-pres-de-4000-policiers-en-moins-d-apres-vanessa-ma] PAS SPECIALEMENT UTILE APRES TOUT
 pour des résultats demandés identiques, voire même supérieurs. Nous pouvons illustrer ceci par les paroles d'Alain Goergen : \textit{Il y a de grosses difficultés budgétaires [\ldots] ce qui implique une optimalisation. Faire aussi bien voire mieux mais avec beaucoup moins de moyens} (Nathalie Delobbe, 2015, \cite{tableronde}). \newline

De plus, du fait des récents évènements terroristes en France, l'armée Belge travaille en coopération avec la police pour assurer la sécurité des citoyens. Le fait d'être passé au niveau d'alerte 3 augmente le coût opérationnel des forces de l'ordre. Ceci demande donc de la coordination et des budgets plus conséquents.

L'année passée Mr Goergen a eu la tâche de réorganiser le département RH. En effet celui-ci n'était pas un département en tant que tel puisqu'il s'agissait d'une des compétences de la direction générale, au même titre que la logistique ou l'ICT. Il a fallu diviser ces différents départements en entités individuelles.\\

En plus de cela, une loi a été votée incitant les services publiques et donc la police à faire un pas vers la décentralisation. Ils ont donc choisi de créer 13 entités, sorte de sous-départements responsables d'une zone géographique. La décentralisation d'un compétence est faite si elle apporte une certaine valeur ajoutée. Une partie de celles-ci restent donc centralisées.\\

Le département RH a donc été divisé en 4 entités:
\begin{itemize}
	\item Pour le recrutement et la sélection
	\item Pour la formation avec l'Académie Nationale de Police
	\item Pour la gestion des carrières
	\item Un service psycho-social et médical\\
\end{itemize}

Alain Goergen a donc mené cette restructuration interne de son département. Il doit maintenant modifier la structure des autres départements de la police.\\

Pour terminer, il s'est définit comme un architecte qui a dû s'assurer du bon fonctionnement des changements, ou encore comme un équilibriste en s'assurant aussi que les changements s'effectuent bien pour les personnes concernées.\\