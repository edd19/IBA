\documentclass[12pt]{article}
%Pour les marges
\usepackage[a4paper]{geometry}
\geometry{hscale=0.75,vscale=0.86,centering} 
\usepackage[utf8]{inputenc}
\usepackage[frenchb]{babel}
\usepackage[T1]{fontenc}
\usepackage{amsmath}
\usepackage{amsfonts}
\usepackage{amssymb} 
\usepackage{graphicx}
\usepackage{lmodern}
\usepackage{numprint}
\usepackage{hyperref}
\usepackage{tabularx}
\usepackage[squaren, Gray]{SIunits}
\usepackage{ulem} %Pour barrer
\usepackage{listings}
\usepackage{enumitem}


\usepackage{newtxtext,newtxmath} % pour la police time

\title{Travail Management}
\author{Aymeric De Cocq, Léonard Julémont}
\date{\today}

\begin{document}

\section{Remise en contexte}


Alain Goergen est le directeur des ressources humaines de la Police Fédérale. Sa carrière s'est construite au sein de la police. Il commence par suivre une formation d'officier de gendarmerie à l'école militaire avec une licence en criminologie. Après 2 ans sur le terrain il entame une seconde formation en management pour se réorienter vers un poste administratif. Il a été en charge de nombreux projets toujours au sein des ressources humaines comme le recrutement, la formation ou encore la restructuration complète du département RH.\\

La Police Fédérale est le plus grand employeur de Belgique. Elle compte 193 employeurs qui emploient plus 50 000 personnes réparties dans différents domaines d'activités sur tout le territoire national. L'éventail de missions de la Police est assez large: maintien de l'ordre, assistance aux victimes, missions anti-terroristes, etc.\\
Malgré la division entre les 192 polices locales et la police fédérale, elles sont toutes dépendantes du département RH de la police fédérale.\\
 
La Police est aujourd'hui dans un phase difficile du point de vue économique et politique. En effet, le gouvernement doit faire des économies et ce service public n'est pas épargné. Le budget de la Police est donc moindre que les années précédentes [\textbf{trouver chiffres?}], pour des résultats identiques, voir même supérieurs. Nous pouvons illustrer ceci par les paroles d'Alain Goergen : 

\begin{center}
	\textit{Il y a de grosse difficulté budgétaire [...] ce qui implique une optimalisation. Faire aussi bien voir mieux mais avec beaucoup moins de moyens.}
\end{center} 

De plus, du fait des récents évènements terroristes en France, l'armée Belge travaille en coopération avec la police pour assurer la sécurité des citoyens. Le fait d'être passé au niveau d'alerte 3 augmente le coût opérationnel des forces de l'ordre. Ceci demande donc de la coordination et des budgets plus conséquents.

\section{Les rôles missions etc.}
L'année passée Mr Goergen a eu la tâche de réorganiser le département RH. En effet celui-ci n'était pas un département en tant que tel puisqu'il s'agissait d'une des compétences de la direction générale, au même titre que la logistique ou l'ICT. Il a fallu diviser ces différents départements en entités individuelles.\\

En plus de cela, une loi a été votée incitant [les services publiques \textbf{( vérifier si les services publics aussi)}] et donc la police à faire un pas vers la décentralisation. Ils ont donc choisi de créer 13 entités, sorte de sous-départements responsables d'une zone géographique. La décentralisation d'un compétence est faite si elle apporte une certaine valeur ajoutée. Une partie de celles-ci restent donc centralisées.\\

Le département RH a donc été divisé en 4 entités:
\begin{itemize}
	\item Pour le recrutement et la sélection
	\item Pour la formation avec l'Académie Nationale de Police
	\item Pour la gestion des carrières
	\item Un service psycho-social et médical
\end{itemize}

Alain Georgen a donc mené cette restructuration interne de son département. Il faut maintenant modifier la structure des autres départements de la police.\\

Il a joué plusieurs rôles dans le cadre de ces restructurations. Il s'est définit comme un architecte qui a dut s'assurer du bon fonctionnement des changements, ou encore comme un équilibriste en s'assurant aussi que les changements s'effectuent bien pour les personnes concernées.

%Il a donc géré toute cette partie de modification de la structure, mais dans un autre temps ils vont devoir modifier la structure des autres départements. 
%Avec tous ces changements, ils doivent faire en sorte que tout fonctionne correctement. Il faut coordonner les gens. 
%Il définit donc son rôle comme = un architecte, un équilibriste, toucher à la structure, operationnaliser, et faire en sorte que les choses fonctionnent. (on pourrait le citer). 


%-->Rentrer plus en détail sur ce qui le caractérise comme 
%Dans ce processus de restructuration, Alain Georgen a identifié deux rôles qu'il doit jouer.\\
% Il doit tout d'abord être \textbf{l'architecte de la restructuration:} il doit imaginer, construire et faire en sorte que tout fonctionne. Le succès de la restructuration commencera par le travail que son équipe aura réalisé sur les emplois à délocaliser ou non. \\
% En parallèle à ce rôle d'architecte, il doit jouer aussi celui \textbf{d'équilibriste}. Les pressions de l'état sont assez fortes, lui demandant d'effectuer des coupes budgétaire importantes. De l'autre il doit ne doit pas oublier les personnes qui sont derrière la Police. Il doit être là pour accompagner les personnes lors de ces bouleversements. Des changements de cette ampleur apportent beaucoup d'incertitudes et de crainte pour les employés. La police est composés de personnes avec de grand nombre de fonctions et formations différentes. Il faut donc arriver à faire en sorte que pour chacun des postes, cette restructuration se fera dans les meilleures conditions possibles.\\
% Pour finir, le troisième rôle qu'il doit endosser est celui \textbf{d'agent de changement}. La restructuration commence par le département des ressources humaines. Il doit en quelques sortes montrer l'exemple sur la voie à prendre. 
 %Je ne vois pas vraiment d'ou vient ce paragraphe. 
 
 %--> J'ai peut-être extrapoler ici. Il a mentionné dans l'interview "C’est aussi accompagner le changement parce qu’on est dans une organisation qui est finalement un peu plus fermé que d’autres. On a un turn-over qui est très très bas par rapport à d’autres entreprises." Mais je suis partis en Live a 2h du matin apparement, on peut supprimer donc ^^
\section{Place dans le modèle d'Ulrich}

%\begin{itemize}
%	\item \textbf{Champion des employés:} Assurer que les membres du personnel fassent ce qui est attendu d'eux, y trouvent satisfaction, puissent se développer et rester impliqués 
%	\item \textbf{Agent du changement:} Assurer que la culture de l'organisation évolue de sorte que les comportements des individus soient bien adaptés à l'environnement 
%	\item \textbf{Expert administratif:} Assurer que les règles, processus, systèmes et outils RH fonctionnent de manière efficace et efficiente.
%	\item \textbf{Partenaire stratégique:} Assurer que le déploiement des ressources humaines aide l'organisation à atteindre ses objectifs.
%\end{itemize}

En lien avec le modèle d'Ulrich, on peut commencer par définir Alain Georgen comme un partenaire stratégique. Un partenaire stratégique doit assurer que le de déploiement des réformes mises en place permettent d'atteindre les objectifs recherchés. Cette restructuration a beaucoup de retombées dans différents domaines. A cause de la baisse de budget accordé à la Police, il a fallu redéfinir avec beaucoup d'attention les compétences à décentraliser ou non, prévoir les répercussions financières de celles-ci etc. Ceci a donc nécessité un travail en amont important.\\

Dans la continuité de cette restructuration interne, on peut définir Alain Goergen comme un agent changement. Un agent du changement est quelqu'un qui s'assure que l'évolution de l'entreprise dans laquelle il travaille se passe correctement, que les comportements des individus soient bien adaptés à l'environnement. C'est exactement ce rôle qu'a joue M. Georgen dans le cadre de cette restructuration. 

%Police très éloignée de tous ses employés, sur tout le territoire, le DRH ne voit que ses collègues dans son bureau. C'est donc difficile d'être proche des employés et de pouvoir s'occuper beaucoup d'eux.\\

%Bon expert administratif car tout contrôlé par l'état. Faire très attention à ses dépenses, il faut être le plus efficace possible d'où la restructuration\\
%Je dirai que c'est le rôle du département en général

%--> C'est vrai! Ca peut peut etre faire l'objet d'une réfléxion générale plus tard dans le projet. Chaque DRH s'est impliqué dans une ou des missions spécifiques au cours de la dernière année, néanmoins tous les rôles d'Ulrich sont important pour un bon fonctionnement des RH et ils sont pris en charge par le département ou d'autres personnes

%Je pense que lui ne l'est pas vraiment mais le département oui parce que toute la partie selection/formation est faite pour trouver les talents dont ils ont besoin. 
%aussi un peu de Champion des employés car ils font en sorte de prendre des gens qui ont la bonne mentalité. 
%Par contre je n'arrive pas a retrouver s'il a parlé du fait qu'ils doivent motiver, ....

%--> Même chose que au dessus, est-ce-qu'on parle du département en général dans la partie? C'estun commentaire qu'on peut faire pour les trois DRH je pense et donc peut-être le mettre plus loins dans le travail.

\end{document}
