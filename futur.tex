\section{Nouvelle fonction RH}

La dernière section de ce travail consiste ajouter un nouveau rôle au modèle théorique, sur base de l'analyse que l'on a pu faire du discours des trois DRH que nous avons pu entendre. Ce rôle ne vient pas remplacer les précédents, mais les compléter.  Nous appellerons ce nouveau rôle ??? \og médiateur \fg{}.\\

Comme nous le savons grâce au modèle théorique d'Ulrich, la fonction RH des sociétés d'aujourd'hui tend vers le \og Partenaire stratégique \fg{}, même si comme nous l'avons vu, un département peut assumer plusieurs rôles. Cette tendance devrait continuer dans le futur, mais sans doute avec quelques petits changements. Nous voyons le médiateur comme une personne toujours proche de la direction mais aussi proche des employés. \\

Attention, par \og médiateur \fg{}, nous ne parlons pas de quelqu'un qui va résoudre des conflits, mais bien quelqu'un qui va permettre le dialogue. Dans la quasi totalité des sociétés, les décisions et idées vont toujours dans le même sens : direction vers employés. Mais cela pourrait très bien changer ! Comme nous l'avons vu, le DRH est devenu proche de la direction et du CEO. Mais il est aussi proche des employés, et ne doit pas perdre ce contact. Mme Thiran et Mr Nolf nous ont dit qu'ils demandaient des idées aux employés. Ils peuvent ensuite bâtir sur ces idées. Tel est le rôle que nous voyons pour le futur du département ressources humaines. \\

Il s'agit donc d'un rôle qui permet aux employés d'exprimer leur voix de manière constructive. Bien souvent, les employés sont plus sur le terrain que la direction et peuvent donc apporter une autre vision de la stratégie que l'entreprise pourrait prendre. Ce rôle se reste toujours proche du \og Partenaire stratégique \fg{} car le but est toujours d'augmenter l'efficacité de la société, d'atteindre les objectifs, ... mais en faisant participer les employés, ce qui leur permettra d'accepter beaucoup plus facilement ce qui est demandé.\\

Le \og Médiateur \fg{} devrait donc communiquer aux employés la culture de l'entreprise, et s'assurer qu'elle est bien présente chez eux. Cela permet par la suite que les idées proposées soient en accord avec la volonté générale de la direction. En effet, il ne faut pas que les idées soient complètement à l'opposé de la culture d'entreprise. La communication est aussi importante dans l'autre sens, ainsi que la motivation des employés, car une fois la solution choisi, il faut la mettre en œuvre. Pour cela, le département RH s'associe avec les managers pour que les employés soient motivés à l'idée d'appliquer une idée à laquelle ils ont participé. On peut dire ici que le DRH doit prendre un rôle de leader. \\ 