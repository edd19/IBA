\subsection{Nouvelle fonction RH}

La dernière section de ce travail consiste ajouter un nouveau rôle au modèle théorique, sur base de l'analyse que l'on a pu faire du discours des trois DRH que nous avons pu entendre. Ce rôle ne vient pas remplacer les précédents, mais le compléter.  Nous appellerons ce nouveau rôle ??? \og médiateur \fg{}.\\

Comme nous le savons grâce au modèle théorique d'Ulrich, la fonction RH des sociétés d'aujourd'hui tend vers le \og Partenaire stratégique \fg{}, même si comme nous l'avons vu, un département peut assumer plusieurs rôles. Cette tendance devrait continuer dans le futur, mais sans doute avec quelques petits changements. Nous voyons le médiateur comme une personne toujours proche de la direction mais aussi proche des employés. \\

Attention, par \og médiateur \fg{}, nous ne parlons pas de quelqu'un qui va résoudre des conflits, mais bien quelqu'un qui va permettre le dialogue. Dans la quasi totalité des sociétés, les décisions et idées vont toujours dans le même sens : direction vers employés. Mais cela pourrait très bien changer ! Comme nous l'avons vu, le DRH est devenu proche de la direction et du CEO. Mais il est aussi proche des employés, et ne doit pas perdre ce contact. Mme Thiran et Mr Nolf nous ont dit qu'ils demandaient des idées aux employés. Ils peuvent ensuite bâtir sur ces idées. Tel est le rôle que nous voyons pour le futur du département ressources humaines. \\