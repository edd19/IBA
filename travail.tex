\documentclass[a4paper, 12pt]{article}

\usepackage[frenchb]{babel} % Document en français
\usepackage[utf8]{inputenc} % Document au format UTF8
\usepackage[T1]{fontenc} % Suppression d'un warning pour la langue francaise
%\usepackage[left=2.2cm, right=2.2cm, top=2.5cm, bottom=2.5cm]{geometry} % Mise en page
\usepackage{setspace} % Permet de définir l'interligne
\usepackage{titling}
\usepackage{graphicx}
\usepackage{todonotes}

\singlespacing % Interligne à 1

\newcommand{\HRule}{\rule{\linewidth}{0.5mm}}
\graphicspath{{res/}}

\usepackage{newtxtext,newtxmath} % pour la police time


\begin{document}

\begin{titlepage}
\begin{center}

% Upper part of the page. The '~' is needed because \\
% only works if a paragraph has started.
\includegraphics[width=0.15\textwidth]{./sedes_ucl.png}~\\[1cm]

\textsc{\LARGE Université Catholique de Louvain}\\[1.5cm]

\textsc{\Large Travail pratique de Management humain}\\[0.5cm]

% Title
\HRule \\[0.4cm]
{ \huge \bfseries Les rôles de la fonction Ressources Humaines\\[0.4cm] }

\HRule \\[1.5cm]

% Author and supervisor
\noindent
\begin{minipage}[t]{0.4\textwidth}
\begin{flushleft} \large
\emph{Auteurs:}\\
Ivan \textsc{Ahad} \\
Aymeric \textsc{De Cocq} \\
Solaiman \textsc{El Jilali} \\
Léonard \textsc{Julémont} \\
Eddy \textsc{Ndizera} \\
Florian \textsc{Thuin}

\end{flushleft}
\end{minipage}%
\begin{minipage}[t]{0.4\textwidth}
\begin{flushright} \large
\emph{Professeurs :} \\
Nathalie \textsc{Delobbe} \\
Emilie \textsc{Malcourant}
\end{flushright}
\end{minipage}

\vfill

% Bottom of the page
{\large \today}

\end{center}
\end{titlepage}

%\maketitle

%\tableofcontents

\section{Présentation des différents conférenciers}
Notre groupe a eu l'occasion de participer à la table ronde accueillant trois directeurs de ressources humaines : Christine Thiran, Frédéric Nolf, Alain Goergen. \newline

La première est Christine Thiran qui travaille actuellement comme directrice des ressources humaines pour le groupe Mestdagh. Durant son parcours, elle a occupé le poste de DRH chez Belgacom durant dix années  où elle a notamment été responsable d'un plan de restructuration concernant 4500 personnes.
Elle a ensuite travaillé pendant 11 ans aux cliniques universitaires de Saint-Luc où elle a mis en place une section ressources humaines.
Depuis cinq mois, elle travaille au sein de cette entreprise familiale du secteur de la grande distribution. Ce secteur en grande difficulté disposant de marges limitées emploie majoritairement du personnel non-qualifié. Le groupe Mestdagh a comme vision d'offrir de l'emploi dans des zones économiquement défavorisées, et doit actuellement intégrer 16 anciennes enseignes (et leurs employés) du groupe Carrefour.
Durant cette courte période, Mme Thiran a émis prématurement un rapport d'étonnement questionnant entre autres les habitudes pour accélérer la prise de conscience des éléments à modifier. \newline

Ce faisant, Mme Thiran s'est identifiée à plusieurs rôles dont la gestion d'intégration des employés des anciennes enseignes Carrefour rachetées. Elle remarqua un clivage persistant entre les employés Carrefour et Mestdagh. Elle doit donc avoir un rôle de \textbf{fédérateur} et rassembler l'ensemble des employés Mestdagh autour d'une vision commune, d'une culture commune. \todo[inline]{Pe a completer} garder l'esprit familial de l'entreprise et sa proximité avec les gens, offrir des perspectives de carrière au sein des différents magasins. Concernant les recrutements, Mestdagh recrute principalement par interim. Mme Thiran s'occupe principalement d'évaluer régulièrement les compétences de ses employés. Elle veillera à ce que les nouveaux employés aient une bonne capacité d'intégration et qu'ils savent garder la bonne humeur. (à modifier si pas assez bien écrit?)\newline

Parallèlement, Mme Thiran se considère comme un \textbf{moteur de changement}. Mestdagh évolue dans un secteur très compétitif et où les marges de bénéfices sont très petites. De ce fait, cela a un impact sur l'organisation et donc sur la gestion des ressources humaines. De ce fait, son rôle est de secouer l'ensemble de ses collaborateurs. Via ce rapport, elle tente d'initier un changement, une évolution de la situation chez Mestdagh. \newline
Pour ce faire, une de ses pratiques est de pousser l'ensemble des employés mais aussi des syndicats à participer, à etre consulté dans les prises de décisions. De plus, elle se voit comme le bras droit du CEO et veut entrer en symbiose avec celui-ci dans les prises de décisions. Elle se trouve ainsi dans une position optimale de par sa proximité avec à la fois le CEO, les employés et les syndicats.\newline

Le second était Frédéric Nolf, responsable des ressources humaines pour IBA depuis 2007. IBA est une société côtée en bourse, basée sur l'innovation. Lors de sa création c'était une spin-off de l'UCL et elle est actuellement encore en alliance avec les universités pour trouver des nouveaux talents. Elle est côtée en bourse. Son CEO a changé il y a trois ans.

Son rôle est d'assurer à l'entreprise qu'elle dispose des bonnes personnes pour son développement et que les personnes soient dans la bonne entreprise pour leur développement. Ce rôle s'articule autour de plusieurs missions : s'assurer de la motivation de son personnel, assurer un \og{} sens du focus\fg{}(nous y reviendrons), créer des liens de compétence à travers les divers départements d'IBA. \newline


Dans le cadre de ces missions, il doit relever de nombreux défis : faire rêver les ingénieurs nouvellement diplomés de travailler chez IBA, s'assurer que ces ingénieurs s'intègrent à la culture IBA, s'assurer que les ingénieurs travaillent réellement sur ce qui leur est demandé (\og{} quand il y a beaucoup de gens très intelligents, chacun a son idée sur ce qu'il faut faire \fg{}), informer les membres du personnel à travers le monde sur les projets actuels pour trouver les membres les plus aptes à les résoudre. \newline

Ces défis nécessitent de mettre en place certaines pratiques de GRH : faire des enquêtes de satisfaction et y répondre de manière concrète, mettre en place des activités communes (sport), prendre en compte toutes les idées proposées, créer des projets qui lient les convictions personnelles du personnel à la culture de la société (mise en place d'une cellule de réflexion pour le développement durable). \newline

Le dernier intervenant est Alain Goergen, directeur des ressources humaines de la Police Fédérale. Sa carrière s'est construite au sein de la police. Il commence par suivre une formation d'officier de gendarmerie à l'école militaire avec une licence en criminologie. Après 2 ans sur le terrain il entame une seconde formation en management pour se réorienter vers un poste administratif. Il a été en charge de nombreux projets toujours au sein des ressources humaines comme le recrutement, la formation ou encore la restructuration complète du département RH.\\

La Police Fédérale est le plus grand employeur de Belgique. Elle compte 193 sections qui emploient plus de 50 000 personnes réparties dans différents domaines d'activités sur tout le territoire national. L'éventail de missions de la Police est assez large: maintien de l'ordre, assistance aux victimes, missions anti-terroristes, etc.\\
Malgré la division entre les 192 polices locales et la police fédérale, elles sont toutes dépendantes du département RH de la police fédérale.\\
 
La Police est aujourd'hui dans un phase difficile du point de vue économique et politique. En effet, le gouvernement doit faire des économies et ce service public n'est pas épargné. Le budget de la Police est donc moindre que les années précédentes %[\textbf{trouver chiffres?} http://www.sudinfo.be/1166936/article/2014-12-10/impact-de-la-baisse-de-budget-pres-de-4000-policiers-en-moins-d-apres-vanessa-ma] PAS SPECIALEMENT UTILE APRES TOUT
 pour des résultats demandés identiques, voire même supérieurs. Nous pouvons illustrer ceci par les paroles d'Alain Goergen : \textit{Il y a de grosses difficultés budgétaires [\ldots] ce qui implique une optimalisation. Faire aussi bien voire mieux mais avec beaucoup moins de moyens} (Nathalie Delobbe, 2015, \cite{tableronde}). \newline

De plus, du fait des récents évènements terroristes en France, l'armée Belge travaille en coopération avec la police pour assurer la sécurité des citoyens. Le fait d'être passé au niveau d'alerte 3 augmente le coût opérationnel des forces de l'ordre. Ceci demande donc de la coordination et des budgets plus conséquents.

L'année passée Mr Goergen a eu la tâche de réorganiser le département RH. En effet celui-ci n'était pas un département en tant que tel puisqu'il s'agissait d'une des compétences de la direction générale, au même titre que la logistique ou l'ICT. Il a fallu diviser ces différents départements en entités individuelles.\\

En plus de cela, une loi a été votée incitant les services publiques et donc la police à faire un pas vers la décentralisation. Ils ont donc choisi de créer 13 entités, sorte de sous-départements responsables d'une zone géographique. La décentralisation d'un compétence est faite si elle apporte une certaine valeur ajoutée. Une partie de celles-ci restent donc centralisées.\\

Le département RH a donc été divisé en 4 entités:
\begin{itemize}
	\item Pour le recrutement et la sélection
	\item Pour la formation avec l'Académie Nationale de Police
	\item Pour la gestion des carrières
	\item Un service psycho-social et médical\\
\end{itemize}

Alain Goergen a donc mené cette restructuration interne de son département. Il doit maintenant modifier la structure des autres départements de la police.\\

Pour terminer, il s'est définit comme un architecte qui a dû s'assurer du bon fonctionnement des changements, ou encore comme un équilibriste en s'assurant aussi que les changements s'effectuent bien pour les personnes concernées.\\

%Comparaison des rôles RH
\section{Comparaison des rôles RH}

Maintenant que nous avons une idée claires des rôles des différents intervenants, nous pouvons entamer une comparaison de ceux-ci. \\

%Comparaison

Pour ce qui est d'Alain Goergen, on voit directement son rôle de gestionnaire du changement à la Police Fédérale. Il s'agit ici d'un rapprochement que l'on peut faire avec Christine Thiran qui doit aussi gérer le changement qui s'opère actuellement chez Mestdagh. Mr Goergen doit restructurer le département RH et ensuite les autres départements, ainsi que gérer les inquiétudes de chacun quant à ce changement. Par contre, dans le cas des magasins Champion, il s'agit d'une fusion avec certains magasins Carrefour. Mme Thiran doit donc gérer les différences d'opinion et les tensions qu'il peut déjà exister entre les deux groupes. \\

La comparaison du rôle de Mr Nolf et de Mr Goergen est beaucoup plus rapide, car ils n'ont pas vraiment de point commun. En effet Mr Nolf ne gère pas vraiment un changement au sein de son entreprise. \\

%Sorte de conclusion de la comparaison
Suite à cette comparaison, on se rend vite compte que chaque DRH a un rôle bien particulier en fonction de son entreprise, mais aussi et surtout en fonction de la conjoncture dans l'entreprise. Nous pouvons encore une fois rappeler que ces DRH ont eu d'autres rôles au cours de leur carrière. 

%Rôles des DRH dans le modèle d'Ulrich
%Comparaison avec modele ulrich
\section{Rôles des DRH dans le modèle d'Ulrich}

Nous avons à présent une vision des différents rôles que les intervenants endossent. Comme nous le savons, le modèle d'Ulrich donne quatre rôles que peut avoir un responsable de ressources humaines. Il peut donc être intéressant de replacer dans ce modèle, les différents DRH que nous avons rencontré. \\

Commençons par Alain Goergen. Il nous a décrit sa mission comme étant la restructuration complète du département des ressources humaines, et par la suite d'autres départements de la Police Fédérale. Mais il doit aussi aider les employés à s'habituer à ce changement. Au vue de ce rôle qu'il doit jouer, on voit très clairement que cela correspond à l'\og agent de changement \fg{} dans le modèle d'Ulrich. Pour rappel, ce dernier a pour but de faciliter le changement, donc faire en sorte que tout le monde accepte et s'adapte au changement.\\

Nous pouvons aussi le voir comme \og partenaire stratégique \fg{}. En effet, la décentralisation qu'il doit opérer, a un impacte stratégique en plus d'un impact sur les employés comme présenté ci-dessus. En reprenant les mots de Mr Goergen \og Le modus était tout ce qui peut être déconcentré doit être déconcentré mais si il n’y a pas de plus value à déconcentrer il faut le garder au niveau central \fg{} \footnote{TO DO : url video + minute}. Il parle ici de \og plus value \fg{}, il y a donc fait une analyse stratégique sur les compétences à décentraliser. Dans ce cas, il ne lui a pas été demandé d'appliquer un plan pensé en amont, mais de le créer par lui même. Ses décisions ont donc été prises en fonction de l'objectif d'efficacité à atteindre. Il est aussi important de se rappeler que la Police fait fasse à des réductions de budget et que ses décisions doivent les prendre en compte. Cela fait donc de lui un \og partenaire stratégique \fg{} au sens d'Ulrich.   


%Conclusion du la comparaison au modèle 

%Limites de la théorie
%Partie 4 A : Prise de recul et discussion des limites de la théorie
\section{Limites de la théorie}

Nous connaissons les missions et les rôles des intervenants que nous avons pu rencontrer, nous avons comparé leurs rôles avec le modèle d'Ulrich mais il apparait clairement que ce modèle n'est plus à jour. Dans cette section, nous nous forcerons donc de détailler ce que nous pensons être les limites de ce modèle. Les différents points que nous aborderons sont des rôles multiples que peut avoir un DRH, du manque de temporalité et pour finir, du département RH. \\


%Rôles multiples du DRH 

Le premier point qui saute aux yeux, lors du rapprochement entre les rôles de ces DRH et le modèle d'Ulrich, est la possibilité d'avoir plusieurs rôles. Comme nous l'avons dit plus haut, Christine Thiran est \og Agent de changement \fg{}, de part sa mission d'accompagner la fusion du groupe Mestdagh avec des magasins Carrefour, mais aussi \og Championne des employés \fg{} \textbf{verifier que c'est bien ca que l'on dit plus haut}. Mr Nolf est quand à lui un \textbf{???}. Ces personnes assument donc plusieurs rôles du modèle théorique. \\

Il s'agit donc ici d'un première limite du modèle d'Ulrich. Chaque rôle présent dans ce modèle correspond à une période de l'histoire avec son contexte économique, une manière de concevoir l'organisation, une conception de la main d'œuvre. Chacun d'eux exprime ce que fait le responsable des ressources humaines dans son entreprise. Ce modèle donne donc un rôles à chaque DRH en fonction de sa mission %a un certain moment
  mais ne permet pas qu'il puisse en avoir plusieurs.\\ 
%Si on énumère les Mestdagh et IBA on peut aussi rappeler où se situe la Police.

La première limite de ce modèle est donc le manque d'une pluralité des fonctions pour un même DRH. En réalité, on sait que nous avons hérité de certaines pratiques de chacun de ces rôles. Et donc il est tout à fait possible qu'aujourd'hui un responsable des ressources humaines assume plusieurs fonctions en même temps dans son métier. \\
%"En réalité, on sait que nous avons hérité de certaines pratiques de chacun de ces rôles." Pas compris, comment on "sait"?
% Pas plutôt dire: Grâce à l'exposé des trois intervenants qu'on a analysé, on a bien vu qu'un responsable des ressources humaines s'inscrivait dans plusieurs rôles à la fois.
%(Léo)-->On le sait parce qu'au cours elle donnait a chaque fois les pratiques hérités d'une fonction RH. Donc aujourd'hui on a hérité des pratiques de chaque fonction. 
%Je dirai donc : En réalité, comme vu au cours, on sait que nous avons hérité de certaines pratiques de chacun des rôles du modèle d'Ulrich.


%Manque de temporalité/dynamisme

Le deuxième point que nous souhaitons aborder %dans les limites de ce modèle d'Ulrich
, est le manque de temporalité, ou encore, de dynamisme dans le modèle. Certes, comme dit précédemment, ce modèle a été construit en fonction de l'évolution du rôle de la fonction RH dans le temps. Mais  par contre, il ne donne pas la possibilité d'une évolution entre les rôles, pour un même DRH. Ce point est proche du précédent. En effet, il s'agit aussi du fait qu'un DRH peut avoir plusieurs rôles. Mais cette fois nous attirons l'attention sur le point de vue dynamique du rôle. \\ %Les missions dans une entreprises varient en permanence. Lors de recensions économiques, le DRH doit plus se préoccuper de sa stratégie pour assurer la pérennité de l'entreprise. En revanche, lors de conflits sociaux au sein de l'entreprise, le DRH doit appuyer son travail sur ses employés.
%(Léo)-->Je vois pas trop pourquoi le DRH s'occupe de la perenité de l'entreprise. 

Les trois intervenants nous ont présenté leur parcours en tant que responsable des ressources humaines. Ces parcours, bien que fort différents, ont un point commun : tous ont eu plusieurs rôles au fil du temps. Prenons par exemple Alain Goergen, qui dans la même entreprise est passé par les ressources humaines dans le recrutement, la formation et aujourd'hui dans la réorganisation complète de son département. On voit donc une évolution de sa mission au cours du temps. Selon le modèle d'Ulrich, son rôle a évolué dans le temps, passant par exemple du rôle d'\og expert administratif \fg{} à celui d'\og agent de changement \fg{}. \\ %super!

Cette évolution du rôle des DRH dans le temps, est un autre manquement au modèle théorique. C'est ce que nous avons appelé ici plus haut le manque de temporalité, ou encore, de dynamisme. Il est donc important de garder à l'esprit que dans une société, une personne a bien souvent un rôle qui évolue au cours du temps, en fonction du contexte économique, social ou opérationnel de son entreprise. \\ %Est-ce-que ce n'est pas ce qui est dis déjà plus haut? --> ou ca plus haut ? 


%Département RH 

Le troisième et dernier point que nous aborderons ici, comme limite de ce modèle théorique, est le département RH en lui même. Depuis quelques temps déjà, les entreprises se sont rendues compte de l'importance de leurs employés, et de leur bien être. Les départements sont apparus et ils prennent une place de plus en plus importante dans les sociétés. Les intervenants nous l'ont dit, ils sont de plus en plus proches de la direction, et leur département est de plus en plus grand, d'un côté parce que la taille des entreprises grandit mais aussi au vu de leur importance. \\ %Une source serait bien pour montrer que les entreprises se rendes compte de l'importance de leurs employés
%(Léo)--> En effet ! 

L'évolution de la taille de ce département, pour quelque raison que ce soit, permet de couvrir plus de missions. Le département peut s'occuper de plus de choses au quotidien. Nous parlons donc encore une fois du fait d'assumer plusieurs rôles à la fois, au sens d'Ulrich. Mais il ne s'agit plus ici d'une seule personne, il s'agit du rôle du département entier, l'ensemble des personnes qui le composent. Le plus bel exemple est sûrement celui de la Police Fédérale. Alain Goergen fait en effet partie d'un département de plus de 500 personnes. Comme il nous l'a présenté, ce département est découpé en entités qui ont chacun un rôle (recrutement, formation, psycho et socio-médical, ...). Ils recouvrent en fait tous les rôles présents dans le modèle d'Ulrich, même si chaque personne n'a peut-être qu'un seul rôle. \\

Ce troisième point découle donc du premier puisqu'il s'agit aussi de la multitude de rôles de la fonction RH, mais cette fois par pour une seule personne, sur le département entier. \\

% Donnez moi votre avis sur ce troisième point, qui est fort proche du premier en réalité. On pourrait peut etre parler du premier et puis simplement dire que le département peut aussi prendre plusieurs roles. En plus si on y reflechi, si une personne du département a plusieurs roles (point 1) le departement a plusieurs roles. Donc je ne sais pas vraiment si ce que j'ai dis est pertinent ou pas :p

%--> Je trouve que c'est intéressant, on parlait des personnes avant et la on fait une généralité sur le département. Moi je garderai :) (Aymeric)


%Conclusion

Enfin, si nous devions recréer un nouveau modèle pour les rôles du DRH, il faudrait surement prendre en compte le trois limites citées ci-dessus. Même si elles ne se basent que sur l'analyse de trois DRH, on peut supposer qu'il en va de même pour beaucoup d'autres. Le modèle d'Ulrich n'est donc plus exactement à jour pour analyser les ressources humaines aujourd'hui, mais il est évident qu'il ne faut pas l'écarter pour autant. Il nous permet de voir comment à évoluer cette fonction dans les entreprises, au cours de l'histoire, mais aussi d'où viennent les pratiques que les responsables de ressources humaines actuels appliquent tous les jours. \\
%Super conclusion

\bibliographystyle{plain}
\bibliography{biblio}

\end{document}
