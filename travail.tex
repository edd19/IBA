\documentclass[a4paper, 12pt]{article}

\usepackage[frenchb]{babel} % Document en français
\usepackage[utf8]{inputenc} % Document au format UTF8
\usepackage[T1]{fontenc} % Suppression d'un warning pour la langue francaise
\usepackage[left=2.2cm, right=2.2cm, top=2.5cm, bottom=2.5cm]{geometry} % Mise en page
\usepackage{setspace} % Permet de définir l'interligne

\singlespacing % Interligne à 1

\author{Ivan \bsc{Ahad} \and Aymeric \bsc{De Cocq} \and Solaiman \bsc{El Jilali} \and Léonard \bsc{Julémont} \and Eddy \bsc{Ndizera} \and Florian \bsc{Thuin}}
\title{Les rôles de la fonction Ressources Humaines}
\date{\today}

\begin{document}
\maketitle

%\tableofcontents

\section{Analyse des rôles selon les discours}
Notre groupe a eu l'occasion de participer à la table ronde accueillant trois directeurs de ressources humaines : Christine Thiran, Frédéric Nolf, Alain Goergen.

Christine Thiran travaille actuellement comme directrice des ressources humaines pour le groupe Mestdagh. Cela fait cinq mois qu'elle est à la tâche dans cette entreprise familiale du secteur de la grande distribution. La grande distribution est un secteur en grande difficulté disposant de marges limitées qui emploie majoritairement du personnel non-qualifié. Le groupe Mestdagh a comme vision d'offrir de l'emploi dans des zones économiquement défavorisées, et doit actuellement intégrer 16 anciennes enseignes (et leurs employés) du groupe Carrefour.

Elle a pour rôle de développer une nouvelle culture d'entreprise qui pourrait convenir au personnel du Groupe Mestdagh et au personnel des magasins Carrefour à intégrer. Ce rôle s'articule autour de plusieurs missions : être l'initiatrice/le moteur du changement, garder l'esprit familial de l'entreprise et sa proximité avec les gens, offrir des perspectives de carrière au sein des différents magasins. Son défi principal est très lié à son rôle puisque c'est de trouver un moyen de mettre d'accord les points de vues divergents et les différentes pratiques des employés des magasins Carrefour avec celles des magasins Mestdagh. Les pratiques mises en place pour y arriver sont entre autres : l'évaluation des compétences, l'écoute des ressentis des travailleurs et de leurs propositions, la concertation syndicale, la coordination avec le CEO.

Frédéric Nolf est responsable des ressources humaines pour IBA depuis 2007. IBA est une société côtée en bourse, basée sur l'innovation. Lors de sa création c'était une spin-off de l'UCL et elle est actuellement encore en alliance avec les universités pour trouver des nouveaux talents. Elle est côtée en bourse. Son CEO a changé il y a trois ans.

Son rôle est d'assurer à l'entreprise qu'elle dispose des bonnes personnes pour son développement et que les personnes soient dans la bonne entreprise pour leur développement. Ce rôle s'articule autour de plusieurs missions : s'assurer de la motivation de son personnel, assurer un \og{} sens du focus\fg{}(nous y reviendrons), créer des liens de compétence à travers les divers départements d'IBA.


Dans le cadre de ces missions, il doit relever de nombreux défis : faire rêver les ingénieurs nouvellement diplomés de travailler chez IBA, s'assurer que ces ingénieurs s'intègrent à la culture IBA, s'assurer que les ingénieurs travaillent réellement sur ce qui leur est demandé (\og{} quand il y a beaucoup de gens très intelligents, chacun a son idée sur ce qu'il faut faire \fg{}), informer les membres du personnel à travers le monde sur les projets actuels pour trouver les membres les plus aptes à les résoudre.

Ces défis nécessitent de mettre en place certaines pratiques de GRH : faire des enquêtes de satisfaction et y répondre de manière concrète, mettre en place des activités communes (sport), prendre en compte toutes les idées proposées, créer des projets qui lient les convictions personnelles du personnel à la culture de la société (mise en place d'une cellule de réflexion pour le développement durable).

Alain Goergen est directeur des ressources humaines pour la police fédérale.

% TODO : Analyse de Alain Goergen

\section{Comparaison des rôles RH}

\end{document}
