\documentclass[a4paper, 12pt]{article}

\usepackage[frenchb]{babel} % Document en français
\usepackage[utf8]{inputenc} % Document au format UTF8
\usepackage[T1]{fontenc} % Suppression d'un warning pour la langue francaise
\usepackage[left=2.2cm, right=2.2cm, top=2.5cm, bottom=2.5cm]{geometry} % Mise en page
\usepackage{setspace} % Permet de définir l'interligne
\usepackage{titling}
\usepackage{graphicx}

\singlespacing % Interligne à 1

\newcommand{\HRule}{\rule{\linewidth}{0.5mm}}
\graphicspath{{res/}}

\usepackage{newtxtext,newtxmath} % pour la police time


\begin{document}

\begin{titlepage}
\begin{center}

% Upper part of the page. The '~' is needed because \\
% only works if a paragraph has started.
\includegraphics[width=0.15\textwidth]{./sedes_ucl.png}~\\[1cm]

\textsc{\LARGE Université Catholique de Louvain}\\[1.5cm]

\textsc{\Large Travail pratique de Management humain}\\[0.5cm]

% Title
\HRule \\[0.4cm]
{ \huge \bfseries Les rôles de la fonction Ressources Humaines\\[0.4cm] }

\HRule \\[1.5cm]

% Author and supervisor
\noindent
\begin{minipage}[t]{0.4\textwidth}
\begin{flushleft} \large
\emph{Auteurs:}\\
Ivan \textsc{Ahad} \\
Aymeric \textsc{De Cocq} \\
Solaiman \textsc{El Jilali} \\
Léonard \textsc{Julémont} \\
Eddy \textsc{Ndizera} \\
Florian \textsc{Thuin}

\end{flushleft}
\end{minipage}%
\begin{minipage}[t]{0.4\textwidth}
\begin{flushright} \large
\emph{Professeurs :} \\
Nathalie \textsc{Delobbe} \\
Emilie \textsc{Malcourant}
\end{flushright}
\end{minipage}

\vfill

% Bottom of the page
{\large \today}

\end{center}
\end{titlepage}

%\maketitle

%\tableofcontents

\section{Analyse des rôles selon les discours}
Notre groupe a eu l'occasion de participer à la table ronde accueillant trois directeurs de ressources humaines : Christine Thiran, Frédéric Nolf, Alain Goergen.

\subsection{Remise en contexte}
Christine Thiran travaille actuellement comme directrice des ressources humaines pour le groupe Mestdagh. Elle était précédemment DRH chez Belgacom durant X années où elle a notamment été responsable d'un plan de restructuration concernant 6000 personnes (voir article). Elle a ensuite travaillé à l'hôpital universitaire de Saint-Luc où elle a mis en place une section de ressources humaines. Cela fait cinq mois qu'elle travaille au sein de cette entreprise familiale du secteur de la grande distribution. Ce secteur en grande difficulté disposant de marges limitées  emploie majoritairement du personnel non-qualifié. Le groupe Mestdagh a comme vision d'offrir de l'emploi dans des zones économiquement défavorisées, et doit actuellement intégrer 16 anciennes enseignes (et leurs employés) du groupe Carrefour. Durant cette courte période, Mme Thiran a émis prématurement un rapport d'étonnement questionnant entre autres les habitudes pour accélérer la prise de conscience des éléments à modifier.

\subsection{Rôles, missions, défis, etc}
Ce faisant, Mme Thiran s'est identifié plusieurs rôles dont la gestion d'integration des employés des anciennes enseignes Carrefour rachetées. Elle remarqua un clivage persistant entre les employés Carrefour et Mestdagh. Elle doit donc avoir un role de \textbf{fédérateur} et rassembler l'ensemble des employés Mestdagh autour d'une vision commune, d'une culture commune. (Pe a completer) garder l'esprit familial de l'entreprise et sa proximité avec les gens, offrir des perspectives de carrière au sein des différents magasins\newline  recrutement?


Parallèlement, Mme Thiran se considère comme un \textbf{moteur de changement}. Mestdagh évolue dans un secteur très compétitif et où les marges de bénéfices sont très petites. De ce fait, cela a un impact sur l'organisation et donc sur la gestion des ressources humaines. De ce fait, son role est de secouer l'ensemble de ses collaborateurs. Via ce rapport, elle tente d'initier un changement, une évolution de la situation chez Mestdagh. \newline
Pour ce faire, une de ses pratiques est de pousser l'ensemble des employés mais aussi des syndicats à participer, à etre consulter dans les prises de décisions. De plus, elle se voit comme le bras droit du CEO et former une symbiose avec celui-ci dans les prises de décisions. (Position optimale car proche du CEO et des employés + syndicats  ?)

Frédéric Nolf est responsable des ressources humaines pour IBA depuis 2007. IBA est une société côtée en bourse, basée sur l'innovation. Lors de sa création c'était une spin-off de l'UCL et elle est actuellement encore en alliance avec les universités pour trouver des nouveaux talents. Elle est côtée en bourse. Son CEO a changé il y a trois ans.

Son rôle est d'assurer à l'entreprise qu'elle dispose des bonnes personnes pour son développement et que les personnes soient dans la bonne entreprise pour leur développement. Ce rôle s'articule autour de plusieurs missions : s'assurer de la motivation de son personnel, assurer un \og{} sens du focus\fg{}(nous y reviendrons), créer des liens de compétence à travers les divers départements d'IBA.


Dans le cadre de ces missions, il doit relever de nombreux défis : faire rêver les ingénieurs nouvellement diplomés de travailler chez IBA, s'assurer que ces ingénieurs s'intègrent à la culture IBA, s'assurer que les ingénieurs travaillent réellement sur ce qui leur est demandé (\og{} quand il y a beaucoup de gens très intelligents, chacun a son idée sur ce qu'il faut faire \fg{}), informer les membres du personnel à travers le monde sur les projets actuels pour trouver les membres les plus aptes à les résoudre.

Ces défis nécessitent de mettre en place certaines pratiques de GRH : faire des enquêtes de satisfaction et y répondre de manière concrète, mettre en place des activités communes (sport), prendre en compte toutes les idées proposées, créer des projets qui lient les convictions personnelles du personnel à la culture de la société (mise en place d'une cellule de réflexion pour le développement durable).

Alain Goergen est directeur des ressources humaines pour la police fédérale.

% TODO : Analyse de Alain Goergen

\section{Comparaison des rôles RH}

\end{document}
