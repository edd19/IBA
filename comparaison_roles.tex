\section{Comparaison des rôles RH}

Maintenant que nous avons établi les rôles des différents intervenants par rapport à leur discours, nous pouvons entamer une comparaison de ceux-ci. \newline

%Comparaison

Pour ce qui est d'Alain Goergen, on voit directement son rôle de gestionnaire du changement à la Police Fédérale. Il s'agit ici d'un rapprochement que l'on peut faire avec Christine Thiran qui doit aussi gérer le changement qui s'opère actuellement chez Mestdagh. Néanmoins, ces deux types de changements sont différents à la Police et chez Mestdagh. M. Goergen doit restructurer le département RH et ensuite les autres départements, ainsi que gérer les inquiétudes de chacun quant à ce changement. Ce changement se fait cependant en incluant les mêmes personnes qui ont déjà adopté la culture organisationnelle de la police. Par contre, dans le cas des magasins Champion, il s'agit d'une fusion avec certains magasins Carrefour. Mme Thiran doit donc gérer les différences d'opinion et les tensions qui peuvent déjà exister entre les deux groupes. \newline

La comparaison du rôle de M. Nolf et M. Goergen est plus simple car ils partagent peu de points communs. En effet, la Police et IBA sont des entreprises bien différentes, qui ont des défis aujourd'hui qui sont bien distincts. IBA se porte bien économiquement ce qui induit qu'elle doit effectuer moins de changements mais plus s'orienter vers une stratégie de croissance alors que la police subit des réductions budgétaires et doit améliorer sa rentabilité avec moins d'effectifs. Ils ne font pas face aux mêmes défis ni au même type de personnel. La même analyse peut se faire entre M. Nolf et Mme Thiran, puisqu'elle travaille actuellement sur un renouvellement de culture organisationnelle, tandis que M. Nolf continue de bâtir une culture organisationnelle déjà existante dans laquelle les nouveaux membres doivent s'intégrer. La première doit recycler deux cultures pour en bâtir une nouvelle, là où le second peut s'appuyer sur une culture existante à améliorer et innover.


%Sorte de conclusion de la comparaison
Suite à cette comparaison, on se rend vite compte que chaque DRH a un rôle bien particulier en fonction de son entreprise, mais aussi et surtout en fonction de la conjoncture de l'entreprise. Nous pouvons encore une fois rappeler que ces DRH ont eu d'autres rôles au cours de leur carrière. 
%Pas compris la dernière phrase.

%Pour la conclusion on pourrait rebondir sur le modèle d'Ulrich:
%On peut voir encore ici une des limite du modèle d'Ulrich que nous avons exposé au paravent: un DRH ne fait pas partie que d'une "case" dans le modèle, ses défis et missions sont souvent le regroupement de plusieurs compétences. Ainsi deux DRH qui appartiennent à la même classe d'Ulrich peuvent tout de même avoir des missions distinctes comme nous l'avon vu entre Alain Goergen et Christine Thiran.