\section{Comparaison des rôles RH}

Maintenant que nous avons une idée claires des rôles des différents intervenants, nous pouvons entamer une comparaison de ceux-ci. \\

%Comparaison

Pour ce qui est d'Alain Goergen, on voit directement son rôle de gestionnaire du changement à la Police Fédérale. Il s'agit ici d'un rapprochement que l'on peut faire avec Christine Thiran qui doit aussi gérer le changement qui s'opère actuellement chez Mestdagh. Néanmoins, ces deux périodes de changements sont différentes à la Police que chez Mestdagh. Mr Goergen doit restructurer le département RH et ensuite les autres départements, ainsi que gérer les inquiétudes de chacun quant à ce changement. Par contre, dans le cas des magasins Champion, il s'agit d'une fusion avec certains magasins Carrefour. Mme Thiran doit donc gérer les différences d'opinion et les tensions qu'il peut déjà exister entre les deux groupes. \\


La comparaison du rôle de Mr Nolf et de Mr Goergen est beaucoup plus rapide, car ils n'ont pas vraiment de point commun. En effet Mr Nolf ne gère pas vraiment un changement au sein de son entreprise. \\

%Aymeric a rajouter: le paragraphe avec Nolf est un peu cash donc je le replacerai par ça:
La comparaison du rôle de Mr Nolf et Mr Goergen est plus simple car ils partagent beaucoup moins de points commun qu'avec Mme Thiran. En effet, la Police et IBA sont des entreprises bien différentes, qui ont des défis aujourd'hui qui sont bien distincts. IBA se porte bien économiquement ce qui induit qu'elle doit effectuer moins de changements mais plus s'orienter vers une stratégie de croissance alors que la Police subit des réductions budgétaires et doit améliorer sa rentabilité avec moins d'effectifs. 


%Sorte de conclusion de la comparaison
Suite à cette comparaison, on se rend vite compte que chaque DRH a un rôle bien particulier en fonction de son entreprise, mais aussi et surtout en fonction de la conjoncture de l'entreprise. Nous pouvons encore une fois rappeler que ces DRH ont eu d'autres rôles au cours de leur carrière. 
%Pas compris la dernière phrase.

%Pour la conclusion on pourrait rebondir sur le modèle d'Ulrich:
%On peut voir encore ici une des limite du modèle d'Ulrich que nous avons exposé au paravent: un DRH ne fait pas partie que d'une "case" dans le modèle, ses défis et missions sont souvent le regroupement de plusieurs compétences. Ainsi deux DRH qui appartiennent à la même classe d'Ulrich peuvent tout de même avoir des missions distinctes comme nous l'avon vu entre Alain Goergen et Christine Thiran.