Maintenant que nous avons une idée claires des rôles des différents intervenants, nous pouvons entamer une comparaison de ceux-ci. \\

%Comparaison

Pour ce qui est d'Alain Goergen, on voit directement son rôle de gestionnaire du changement à la Police Fédérale. Il s'agit ici d'un rapprochement que l'on peut faire avec Christine Thiran qui doit aussi gérer le changement qui s'opère actuellement chez Mestdagh. Mr Goergen doit restructurer le département RH et ensuite les autres départements, ainsi que gérer les inquiétudes de chacun quant à ce changement. Par contre, dans le cas des magasins Champion, il s'agit d'une fusion avec certains magasins Carrefour. Mme Thiran doit donc gérer les différences d'opinion et les tensions qu'il peut déjà exister entre les deux groupes. \\

La comparaison du rôle de Mr Nolf et de Mr Goergen est beaucoup plus rapide, car ils n'ont pas vraiment de point commun. En effet Mr Nolf ne gère pas vraiment un changement au sein de son entreprise. \\

%Sorte de conclusion de la comparaison
Suite à cette comparaison, on se rend vite compte que chaque DRH a un rôle bien particulier en fonction de son entreprise, mais aussi et surtout en fonction de la conjoncture dans l'entreprise. Nous pouvons encore une fois rappeler que ces DRH ont eu d'autres rôles au cours de leur carrière. 