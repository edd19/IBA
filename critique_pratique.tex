%Critique au niveau de la pratique 
\subsection{Réflexions sur le discours des DRH}

La première chose sur laquelle nous aimerions attirer l'œil du lecteur est le fait que nous avons eu le discours de DRH mais pas des personnes qui les entourent. Nous n'avons donc eu qu'un seul point de vue, mais ce n'est pas ce qui nous a le plus dérangé. En effet, nous avons eu le point de vue de la personne qui agit, en l'occurrence des DRH, mais jamais celui de la personne sur qui cet agissement a un impact. Ils ne nous ont pas parlé des retombées de leurs divers missions. Nous ne connaissons que ce qu'ils ont fait mais en aucun cas, ce que les gens en pensent, que ce soit en bien ou en mal. Prenons comme exemple la restructuration du département RH à la Police, on sait qu'elle a été faite, néanmoins on ne sait pas du tout ce que les employés concernés ont pensé du processus de changement, comment ils l'ont accepté, etc. \newline 

Le deuxième questionnement que nous nous sommes fait à la suite de cette conférence concerne l'avis des employés sur le rapprochement des DRH avec la direction. En effet, les DRH présents nous ont parlé du fait qu'ils sont très proches de la direction. Comme dit plus haut, certains ont un rôle de \og{}Partenaire stratégique\fg{}. Ce rôle leur demande d'être proche de la direction, de penser efficacité, stratégie,\ldots et donc peut-être de s'écarter de la mission initiale du DRH, c'est-à-dire d'être proche des employés.
 Le département continue peut-être à effectuer cette mission mais qu'en est-il du DRH, et qu'en pensent les employés ? Omission, oubli ou manque de temps,\ldots Nous n'avons pas eu cette réponse lors de la conférence. \newline

La troisième réflexion concerne les pratiques des DRH. Par exemple nous savons que Mme Thiran essaie d'être proche des employés, mais elle ne va plus loin. En effet, elle pense que les idées pour améliorer le fonctionnement de l'entreprise doivent être construites tous ensemble. Cela implique de consulter les différents partis intéressés pour constituer une solution qui peut tous les satisfaire. \\
Dans le discours de M. Nolf, nous avons aussi pu remarquer ce phénomène de partage d'idées entre les employés et le département RH.\\
Cette concertation est un changement de pratique par rapport au modèle théorique. Nous aborderons ce sujet plus en détail dans la dernière partie de notre travail.\newline

Lors de cette conférence, nous avons eu l'impression, aux dires de Monsieur Goergen, que la Police Fédérale est l'entreprise dans laquelle le département RH dans son ensemble se rapproche le plus du modèle d'Ulrich, par rapport aux deux autres entreprises. Ou encore, que les deux autres entreprises s'en écartent plus en mettant en place la nouvelle nouvelle pratique énoncée ci-dessus.\\
Par contre, la Police essaie de rendre les processus RH du modèle les plus efficaces possibles, le meilleur exemple est son processus recrutement avec un système très perfectionné. On peut se demander pourquoi la Police ne se détache pas autant du modèle théorique que les deux autres entreprises.\\
Bien sûr, plusieurs réponses sont possibles. La première est simplement que M. Goergen ne nous en a pas parlé. Une autre réponse pourrait être le fait que la Police est un service public. De ce fait, le changement pourrait y être plus compliqué et la structure de l'entreprise plus rigide.\newline