%Critique au niveau de la pratique 

- Ils ont tous l'air de dire qu'ils sont proches de la direction, logique puisqu'ils sont les DRH. Ils semblent tous etre partenaire strategique. Mais par contre aucun d'eux ne nous parle du fait de l'efficacité comme le modele (revoir pour etre sur qu'il n'y a que cela). Soit ils ne nous disent pas ce qu'ils font pour etre efficace soit il y a un nouveau role. 
-Le DRH d'IBA, Frédéric Nolf, a tenu un discours plus d'un manager qu'un DRH. Ils parlent souvent sur la société en général et ce n'est pas très clair d'entrevoir clairement son rôle dans l'entreprise.
-Ils ne nous ont pas parlé de conflits en interne qui se seraient passés ou bien de situations mal gérées de leur part. A entendre leurs discours, on a le sentiment que certes le rôle de DRH comporte son lots de difficultés et de défis mais pas de conflits ou de bras de fer. Vision bisounours quoi.
-Ils ne parlent pas de ce qu'ils font au quotidien. Est-ce eux qui gèrent les contrats ou bien ils délèguent ceci ? Si oui, est-ce tout les contrats ou bien seulement de quelques personnes privilégiées (les chercheurs les plus réputés chez IBA par exemple). De même pour les licenciements, embauches, etc. Beaucoup de questions sans réponses.
-