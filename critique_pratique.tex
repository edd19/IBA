%Critique au niveau de la pratique 
\subsection{Autres réflexions}

La première chose sur laquelle nous aimerions attirer l'œil du lecteur est le fait que nous avons eu le discours de DRH mais pas des personnes qui les entours. On n'avons donc eu qu'un seul point de vue, mais ce n'est pas ce que nous a le plus dérangé. En effet, nous avons eu le point de vue de la personne qui agit, en l'occurrence des DRH, mais jamais celui de la personne sur qui cet agissement a un impact. Ils ne nous ont pas parlé des retombés de leur divers missions. Nous ne connaissons que ce qu'ils ont fait mais en aucun cas, ce que les gens en pensent, que ce soit en bien ou en mal. \\ 

La deuxième réflexion que nous nous sommes faite à la suite de cette conférence est ???. En effet, les DRH présents nous ont parlé du fait qu'ils sont très proches de la direction. Comme dit plus haut, certains ont un rôles de partenaire stratégique \textbf{TOUS ?}. Par contre, ils ne parlent pas de ce que pensent les autres employés de ce rôle qu'ils assument. Ce rôle leur demande d'être proche de la direction, de penser efficacité, stratégie, ... et donc peut être de s'écarter de la mission initiale du DRH. Le département a continue peut-être à effectuer cette mission mais qu'en est-il du DRH, et qu'est-ce qu'en pensent les employés ? Nous n'avons pas eu cette réponse lors de la conférence. \\

D'un autre côté, même si nous n'avons pas la version des employés, nous connaissons la façon dont Mme Thiran reste proche des employés. Pour elle, la solution doit être construite tous ensemble. Donc elle ne pense pas pouvoir tout résoudre toute seule. D'ailleurs, pour apaiser les tensions actuelles dans la fusion que connait le groupe Mestdagh, elle demande aux employés quelles sont leur idées pour améliorer la situation. Elle ne nous a pas parlé de résultats, mais nous pouvons voir une nouvelle façon de travailler en tant que partenaire stratégique. \\

Dans le discours de Mr Nolf, nous avons aussi pu remarqué ce phénomène de partage d'idée entre les employés et le département RH \textbf{VERIFIER LES termaes exactes de ce qu'il a expliqué + exemple}. \\

Lors de cette conférence, nous avons eu l'impression, aux dires de Monsieur Goergen, que la Police Fédérale est l'entreprise dans laquelle le département RH dans son ensemble se rapproche le plus du modèle d'Ulrich, par rapport au deux autres entreprises. Ou bien, que les deux autres entreprises s'en écartent plus, même si elles en gardent les missions. Par contre, la Police essaie de rendre ces processus RH les plus efficaces possibles, comme par exemple pour le recrutement avec un système très perfectionné \textbf{on pourrait citer le moment dans la conf ou la prof parle du systeme}. On peut se demander pourquoi la Police ne s'écarte pas vraiment du modèle théorique et ne semble pas faire naitre de nouveau rôle pour la fonction RH. Bien sûr plusieurs réponses sont possibles, la première est peut-être simplement que Mr Goergen ne nous en a pas parlé, au vu de la durée de la conférence. Une autre réponse est le fait que la Police est un service public, et que de ce fait, les rôles restent plus "authentiques" \textbf{je ne sais pas comment le dire, mais je veux dire qu'ils ne s'écartent pas encore vraiment de ce qui a été acquis}. Nous n'avons pas la réponse à cette question, mais l'analyse d'autres entreprises privées et services publics nous permettrait peut-être d'y répondre. \\


- Ils ont tous l'air de dire qu'ils sont proches de la direction, logique puisqu'ils sont les DRH. Ils semblent tous etre partenaire strategique. Mais par contre aucun d'eux ne nous parle du fait de l'efficacité comme le modele (revoir pour etre sur qu'il n'y a que cela). Soit ils ne nous disent pas ce qu'ils font pour etre efficace soit il y a un nouveau role. 


-Le DRH d'IBA, Frédéric Nolf, a tenu un discours plus d'un manager qu'un DRH. Ils parlent souvent sur la société en général et ce n'est pas très clair d'entrevoir clairement son rôle dans l'entreprise. En effet, ils parlent du nombre de réacteurs de proton thérapie vendue, de parts de marché, comment l'entreprise se finance,.... Beaucoup d'informations sur la société en général mais pas grand chose sur ce que fait le DRH. On peut regretter donc le fait que Nolf ne parle pas de ce qu'il fait au quotidien. Est-ce lui qui gère les embauches, les contrats, les licenciements ? Ou bien ils délèguent tout ça à son équipe ? Si c'est lui qui gère, le fait-il pour tout le membre du personnel ou bien quelques cadres (expertes reconnus mondialement). 
 A cela s'ajoute le fait que Frédéric Nolf donne une image d'IBA comme une société qui fait rêver les chercheurs.
 Je cite:
 "fleuron de l'économie belge"
 "On est probablement parmi ce qui se fait de technologique au niveau médical"
 "très internationale  et très basée sur du talent scientifique et très basé sur l'innovation"
 Beaucoup de publicités donc pour IBA.
 
 
 
-Les différents DRH présents ne nous ont pas parlé de conflits en interne qui se seraient passés ou bien de situations mal gérées de leur part. Est-il possible d'arriver à toujours faire des compromis qui satisfait tout le monde. N'y a t il pas , durant la vie de DRH, des conflits qui nécessitent qu'un des 2 partis quittent la société ? Qu'en-est-il des syndicats d'ailleurs chez IBA ? N'ont-ils aucune influence. A entendre leurs discours, on a le sentiment que certes le rôle de DRH comporte son lots de difficultés et de défis mais pas de conflits ou de bras de fer. Vision bisounours quoi.
