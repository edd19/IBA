\documentclass[12pt]{report}
\usepackage[french]{babel}
\usepackage[T1]{fontenc}
\usepackage[utf8]{inputenc}
\title{Rapport Management Humain}
\author{Ndizera Eddy} 
\begin{document}
\maketitle

IBA est une société belge connue pour sa protontérapie (traitement de cancer par radiation). C'est une société internationale disposant d'un capital humain assez important (majoritairement des chercheurs). IBA est axée sur l'innovation, le désir de faire avancer la science.\\

C'est dans ce contexte-là que travaille Frédéric Nolf, GRH d'IBA. Engagé initialement pour s'occuper de la division protonthérapie, il s'est vu confier de nouvelles responsabilités jusqu'à aujoud'hui où il s'occupe de l'entiéreté des ressources humaines de la société. Cela comprend notamment la gestion des embauches. Chez IBA, on embauche des gens engagés et par engagés, ils entendent des gens qui ont un avis positif sur la société. Pour ce faire, ils disposent d'enquête d'engagement pour mesurer si la personne est positif par rapport à son job et s'il a l'intention de rester.\\

Il, le GRH d'IBA, essaye également de garder les gens motivés au travail. C'est ainsi qu'ils essayent de faire du personnel des acteurs et non des spectateurs, de les impliquer dans la vie de la société. Je citerais pour cela l'exemple du comité composé d'employés pour  aider la société à diminuer son empreinte écologique. Un autre exemple est le fait que les employés sont invités, si ils voient des difficultés, à venir en discuter mais aussi à proposer des alternatives. Tout ceci concourt à rendre l'employé impliqué dans la société. \\

Frédéric Nolf, en qualité de GRH, a rencontré plusieurs défis. Un a été de remettre en route le nouveau CEO d'IBA et de mettre en place autour de lui une dream team. Il y a eu aussi une période de refocalisation de la société qui s'était trop diversifié. Ils ont du revoir la structure de la société, rendre plus visible les rôles de certains du département, etc. Durant cette période , l'humain n'a pas eu toujours sa place. Mais un gros défi que rencontre IBA est l'internationalisation de la société. En effet, la société IBA fait 0 \% de son chiffre d'affaire en Belgique. Le gros souci de cette internationalisation est d'adapter la société aux différentes cultures et législations propres à chaque pays. Cela pose également le souci de pouvoir exploiter les compétences de chaque employé. Un employé X en Inde peut disposer de compétences précieuses pour aider à un projet de recherche à Louvain-la-Neuve. Comment exploiter cela. Ceci est un défi majeur. Il n'y a rien de pire pour un employé que de ne pas pouvoir utiliser ses compétences selon Nolf.\\

A ce souci d'internationalisation s'ajoute celui d'être plus productif, et donc de trouver du personnel qualifié. Or ce n'est pas facile de trouver des ingénieurs qualifiés dans le domaine et encore moins de les garder. Les formations apportent une solution mais la durée de 18 à 24 mois posent souci.\\ 

En tant que DRH, Nolf se profile comme quelqu'un qui doit aider l'organisation à devenir capable de faire ce qu'elle doit faire. Il se profile comme un architecte et un transformateur d'entreprise.




\end{document}
