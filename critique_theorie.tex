%Partie 4 A : Prise de recul et discussion des limites de la théorie
\section{Limites de la théorie}

Nous connaissons les missions et les rôles des intervenants que nous avons pu rencontrer, nous avons comparé leurs rôles avec le modèle d'Ulrich mais il apparait clairement que ce modèle n'est plus à jour. Dans cette section, nous nous forcerons donc de détailler ce que nous pensons être les limites de ce modèle. Les différents points que nous aborderons sont des rôles multiples que peut avoir un DRH, du manque de temporalité et pour finir, du département RH. \\


%Rôles multiples du DRH 

Le premier point qui saute aux yeux, lors du rapprochement entre les rôles de ces DRH et le modèle d'Ulrich, est la possibilité d'avoir plusieurs rôles. Comme nous l'avons dit plus haut, Christine Thiran est \og Agent de changement \fg{}, de part sa mission d'accompagner la fusion du groupe Mestdagh avec des magasins Carrefour, mais aussi \og Championne des employés \fg{} \textbf{verifier que c'est bien ca que l'on dit plus haut}. Mr Nolf est quand à lui un \textbf{???}. Ces personnes assument donc plusieurs rôles du modèle théorique. \\

Il s'agit donc ici d'un première limite du modèle d'Ulrich. Chaque rôle présent dans ce modèle correspond à une période de l'histoire avec son contexte économique, une manière de concevoir l'organisation, une conception de la main d'œuvre. Chacun d'eux exprime ce que fait le responsable des ressources humaines dans son entreprise. Ce modèle donne donc un rôles à chaque DRH en fonction de sa mission %a un certain moment
  mais ne permet pas qu'il puisse en avoir plusieurs.\\ 
%Si on énumère les Mestdagh et IBA on peut aussi rappeler où se situe la Police.

La première limite de ce modèle est donc le manque d'une pluralité des fonctions pour un même DRH. En réalité, on sait que nous avons hérité de certaines pratiques de chacun de ces rôles. Et donc il est tout à fait possible qu'aujourd'hui un responsable des ressources humaines assume plusieurs fonctions en même temps dans son métier. \\
%"En réalité, on sait que nous avons hérité de certaines pratiques de chacun de ces rôles." Pas compris, comment on "sait"?
%Pas plutôt dire: Grâce à l'exposé des trois intervenants qu'on a analysé, on a bien vu qu'un responsable des ressources humaines s'inscrivait dans plusieurs rôles à la fois.
%(Léo)-->On le sait parce qu'au cours elle donnait a chaque fois les pratiques hérités d'une fonction RH. Donc aujourd'hui on a hérité des pratiques de chaque fonction. 
%Je dirai donc : En réalité, comme vu au cours, on sait que nous avons hérité de certaines pratiques de chacun des rôles du modèle d'Ulrich.
%--> (aym)Si on sait avoir une citation de son slide ou syllabus ça serait bien, ça fait une référence du coup :)


%Manque de temporalité/dynamisme

Le deuxième point que nous souhaitons aborder %dans les limites de ce modèle d'Ulrich
, est le manque de temporalité, ou encore, de dynamisme dans le modèle. Certes, comme dit précédemment, ce modèle a été construit en fonction de l'évolution du rôle de la fonction RH dans le temps. Mais  par contre, il ne donne pas la possibilité d'une évolution entre les rôles, pour un même DRH. Ce point est proche du précédent. En effet, il s'agit aussi du fait qu'un DRH peut avoir plusieurs rôles. Mais cette fois nous attirons l'attention sur le point de vue dynamique du rôle. \\ %Les missions dans une entreprises varient en permanence. Lors de recensions économiques, le DRH doit plus se préoccuper de sa stratégie pour assurer la pérennité de l'entreprise. En revanche, lors de conflits sociaux au sein de l'entreprise, le DRH doit appuyer son travail sur ses employés.
%(Léo)-->Je vois pas trop pourquoi le DRH s'occupe de la perenité de l'entreprise. 

%(aym) --> C'est peut-être pas le bon mot mais tout c'était pour dire que quand il y a des problèmes d'argent c'est un peu le DRH qui devra regarder si il y a pas des personnes en trop qu'il faudrait liscencier

Les trois intervenants nous ont présenté leur parcours en tant que responsable des ressources humaines. Ces parcours, bien que fort différents, ont un point commun : tous ont eu plusieurs rôles au fil du temps. Prenons par exemple Alain Goergen, qui dans la même entreprise est passé par les ressources humaines dans le recrutement, la formation et aujourd'hui dans la réorganisation complète de son département. On voit donc une évolution de sa mission au cours du temps. Selon le modèle d'Ulrich, son rôle a évolué dans le temps, passant par exemple du rôle d'\og expert administratif \fg{} à celui d'\og agent de changement \fg{}. \\ 

Cette évolution du rôle des DRH dans le temps, est un autre manquement au modèle théorique. C'est ce que nous avons appelé ici plus haut le manque de temporalité, ou encore, de dynamisme. Il est donc important de garder à l'esprit que dans une société, une personne a bien souvent un rôle qui évolue au cours du temps, en fonction du contexte économique, social ou opérationnel de son entreprise. \\ %Est-ce-que ce n'est pas ce qui est dis déjà plus haut? --> ou ca plus haut ? 
%--> C'est la même chose que le "deuxièmme point" pour moi.


%Département RH 

Le troisième et dernier point que nous aborderons ici, comme limite de ce modèle théorique, est le département RH en lui même. Depuis quelques temps déjà, les entreprises se sont rendues compte de l'importance de leurs employés, et de leur bien être. Les départements sont apparus et ils prennent une place de plus en plus importante dans les sociétés. Les intervenants nous l'ont dit, ils sont de plus en plus proches de la direction, et leur département est de plus en plus grand, d'un côté parce que la taille des entreprises grandit mais aussi au vu de leur importance. \\ %Une source serait bien pour montrer que les entreprises se rendes compte de l'importance de leurs employés
%(Léo)--> En effet ! 

L'évolution de la taille de ce département, pour quelque raison que ce soit, permet de couvrir plus de missions. Le département peut s'occuper de plus de choses au quotidien. Nous parlons donc encore une fois du fait d'assumer plusieurs rôles à la fois, au sens d'Ulrich. Mais il ne s'agit plus ici d'une seule personne, il s'agit du rôle du département entier, l'ensemble des personnes qui le composent. Le plus bel exemple est sûrement celui de la Police Fédérale. Alain Goergen fait en effet partie d'un département de plus de 500 personnes. Comme il nous l'a présenté, ce département est découpé en entités qui ont chacun un rôle (recrutement, formation, psycho et socio-médical, ...). Ils recouvrent en fait tous les rôles présents dans le modèle d'Ulrich, même si chaque personne n'a peut-être qu'un seul rôle. \\

Ce troisième point découle donc du premier puisqu'il s'agit aussi de la multitude de rôles de la fonction RH, mais cette fois par pour une seule personne, sur le département entier. \\

% Donnez moi votre avis sur ce troisième point, qui est fort proche du premier en réalité. On pourrait peut etre parler du premier et puis simplement dire que le département peut aussi prendre plusieurs roles. En plus si on y reflechi, si une personne du département a plusieurs roles (point 1) le departement a plusieurs roles. Donc je ne sais pas vraiment si ce que j'ai dis est pertinent ou pas :p

%--> Je trouve que c'est intéressant, on parlait des personnes avant et la on fait une généralité sur le département. Moi je garderai :) (Aymeric)


%Conclusion

Enfin, si nous devions recréer un nouveau modèle pour les rôles du DRH, il faudrait surement prendre en compte le trois limites citées ci-dessus. Même si elles ne se basent que sur l'analyse de trois DRH, on peut supposer qu'il en va de même pour beaucoup d'autres. Le modèle d'Ulrich n'est donc plus exactement à jour pour analyser les ressources humaines aujourd'hui, mais il est évident qu'il ne faut pas l'écarter pour autant. Il nous permet de voir comment à évoluer cette fonction dans les entreprises, au cours de l'histoire, mais aussi d'où viennent les pratiques que les responsables de ressources humaines actuels appliquent tous les jours. \\
%Super conclusion