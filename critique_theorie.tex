%Partie 4 A : Prise de recul et discussion des limites de la théorie
\subsection{Limites de la théorie}

Commençons donc par détailler ce que nous pensons être les limites de ce modèle. Les différents points que nous aborderons sont des rôles multiples que peut avoir un DRH, du manque de temporalité et pour finir, du département RH.\newline

\subsubsection*{Multiplicité des rôles} 

Le premier point qui saute aux yeux, lors du rapprochement entre les rôles de ces DRH et le modèle d'Ulrich, est la possibilité d'avoir plusieurs rôles. \newline

En se basant sur son discours, on observe que Christine Thiran correspond à l'\og{}agent de changement\fg{}, de par sa mission d'accompagnement de la fusion du groupe Mestdagh avec des magasins Carrefour, mais aussi comme \og{}championne des employés\fg{}. \newline

M. Nolf dispose lui aussi de plus d'un rôle. Le premier qu'on peut retrouver est d'être le \og{}champion des employés\fg{}. Il a pour rôle de répondre aux besoins et aux préoccupations des employés mais également d'assurer leur motivation et leur contribution aux objectifs de l'entreprise. Parallèlement, on peut aussi l'associer au rôle de \og{}partenaire stratégique \fg{} puisqu'il a pour mission de trouver les talents de demain et d'assurer que ceux déjà présents permettent à l'organisation d'atteindre ses objectifs, en collaboration avec les autres dirigeants. Dans une moindre mesure (en tout cas, moins discutée lors de la conférence), il est également \og{}agent de changement\fg{} puisqu'il fait évoluer les comportements au sein de l'entreprise pour qu'ils collent à la culture IBA et également qu'IBA suive les avancées technologiques de manière continue. 

Ces personnes assument donc plusieurs rôles du modèle théorique. \\

Il s'agit donc ici d'une première limite du modèle d'Ulrich. Chaque rôle présent dans ce modèle correspond à une période de l'histoire avec son contexte économique, une manière de concevoir l'organisation, une conception de la main d'\oe{}uvre. Chacun d'eux exprime ce que fait le responsable des ressources humaines dans son entreprise. Ce modèle donne donc un rôle à chaque DRH en fonction de sa mission, à un certain moment donné, mais ne permet pas qu'il puisse en avoir plusieurs. Nous modérons tout de même cette affirmation en signalant que Dave Ulrich lui-même considère aujourd'hui (lors de ses conférences), qu'un leader est naturellement bon dans 2 des 4 cadrans, mais le travail aujourd'hui est de le faire devenir bon dans les 4 cadrans. (Andrée Laforge, 2013, \cite{capitalhumain})\newline 

\todo[inline]{Nous modérons? Nous modérons quoi? Tu as changé ce que tu pensais mais ce n'est pas pour ça qu'on modère ce qu'il y avait avant. On peut dire "attention, Ulrich a changé son point de vue, De plus il faut l'URL dans la source et je ne sais pas si on peut utiliser cette source comme ça, c'est un résumé d'une conférence sur un blog}

La première limite de ce modèle est donc le manque d'une pluralité des fonctions pour un même DRH. En réalité, on sait que nous avons hérité de certaines pratiques de chacun des rôles du modèle d'Ulrich (Nathalie Delobbe, 2015,\cite{slidegrh}). Il est donc tout à fait possible qu'aujourd'hui un responsable des ressources humaines assume plusieurs fonctions en même temps dans son métier.
\footnote{Comme cela a été fait dans une étude qui assignait un pourcentage de chaque rôle en fonction du temps sur plusieurs périodes par Lawler, Boudreau \& Mohrman (Lawler, 2006, \cite{Lawler2006})}\\



\subsubsection*{Historicité des rôles}

Le deuxième point que nous souhaitons aborder dans les limites de ce modèle d'Ulrich, est le manque de temporalité, ou encore, de dynamisme dans le modèle. Certes, comme dit précédemment, ce modèle a été construit en fonction de l'évolution du rôle de la fonction RH dans le temps. Mais, par contre, il n'exprime pas la possibilité d'une évolution entre les rôles, pour un même DRH. Ce point est proche du précédent. En effet, il s'agit aussi du fait qu'un DRH puisse avoir plusieurs rôles. Mais cette fois nous attirons l'attention sur le point de vue dynamique du rôle. \\

Les trois intervenants nous ont présenté leur parcours en tant que responsable des ressources humaines. Ces parcours, bien que fort différents, ont un point commun : tous ont eu plusieurs rôles au fil du temps. Prenons par exemple Alain Goergen, qui dans la même entreprise est passé par les ressources humaines dans le recrutement, la formation et aujourd'hui dans la réorganisation complète de son département. On voit donc une évolution de sa mission au cours du temps. Selon le modèle d'Ulrich, son rôle a évolué dans le temps, passant par exemple du rôle d'\og{}expert administratif\fg{} à celui d'\og{}agent de changement \fg{}. \newline 

Cette évolution du rôle des DRH dans le temps, est un autre manquement au modèle théorique. C'est ce que nous avons appelé ici plus haut le manque de temporalité, ou encore, de dynamisme. Il est donc important de garder à l'esprit que dans une société, une personne a bien souvent un rôle qui évolue au cours du temps, en fonction du contexte économique, social ou opérationnel de son entreprise. \\ 


\subsubsection{Du rôle du département RH}

Le troisième et dernier point que nous aborderons ici, comme limite de ce modèle théorique, est le département RH en lui même. Depuis quelques temps déjà, les entreprises se sont rendues compte de l'importance de leurs employés, et de leur bien-être. Les départements sont apparus et ils prennent une place de plus en plus importante dans les sociétés. Les intervenants nous l'ont dit, ils sont de plus en plus proches de la direction et leur département est de plus en plus grand, d'un côté parce que la taille des entreprises croît mais aussi au vu de l'importance qui y est accordée. \newline
\todo[inline]{Une source serait bien pour montrer que les entreprises se rendent compte de l'importance de leurs employés}

L'évolution de la taille de ce département, pour quelle raison que ce soit, permet de couvrir plus de missions. Le département peut s'occuper de plus de choses au quotidien. Nous parlons donc encore une fois du fait d'assumer plusieurs rôles à la fois, au sens d'Ulrich. Mais il ne s'agit plus ici d'une seule personne, il s'agit du rôle du département entier, l'ensemble des personnes qui le composent. Le plus bel exemple est sûrement celui de la Police Fédérale. Alain Goergen fait en effet partie d'un département de plus de 500 personnes. Comme il nous l'a présenté, ce département est découpé en entités qui ont chacune un rôle (recrutement, formation, psycho et socio-médical, ...). Ils recouvrent en fait tous les rôles présents dans le modèle d'Ulrich, même si chaque personne n'a peut-être qu'un seul rôle. \newline

Ce troisième point découle donc du premier puisqu'il s'agit aussi de la multitude de rôles de la fonction RH, cette fois pas pour une seule personne, mais sur le département entier. \newline

%Conclusion
Enfin, si nous devions recréer un nouveau modèle pour les rôles du DRH, il faudrait surement prendre en compte les trois limites citées ci-dessus. Même si elles ne se basent que sur l'analyse de trois DRH, on peut supposer qu'il en va de même pour beaucoup d'autres. Le modèle d'Ulrich n'est donc plus exactement à jour pour analyser les ressources humaines d'aujourd'hui, mais il est évident qu'il ne faut pas l'écarter pour autant car il permet facilement d'avoir une visibilité du positionnement et de l'action des RH. Il nous permet également de voir comment a évolué cette fonction dans les entreprises, au cours de l'histoire, mais aussi d'où viennent les pratiques que les responsables de ressources humaines actuels appliquent tous les jours. \\
