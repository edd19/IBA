%Role du DRH pour l'avenir
	<This is a shaft>
	
	-Le rôle du DRH pour l'avenir ne serait plus, comme dans le modèle d'Ulrich, limité à un rôle (partenaire stratégique, champion des employés, expert administratif ou agent de changement). Il pourrait endosser plusieurs rôles d'Ulrich en même temps ou selon les situations. Il y a donc le fait que le DRH s'adapte à la situation du jour. Il endosse le comportement le plus adapté à une situation. On aurait donc un rôle de DRH plus flexible. 
	Citation d'IBA pour argumenter cela:
	"On a revu la  structure, on a redonné une 
visibilité aux gens de la structure sur les responsabilités, sur le mode de fonctionnement et on a passé 
quelques années à essayer de mettre quelque chose à flot et à cette occasion ça laisse  pas toujours 
autant de place que l'on souhaiterait à l'humain"
(Plus champion des employés à ce moment)

	-De plus, le DRH, de ce qu'on a entendu à la conférence, se rapproche de plus en plus de la direction (du CEO). 
	"Bras droit du CEO" (Madame Thiran)
	Il disposera donc d'une réelle influence sur la société dans laquelle il exerce.  
	
	-Un point commun à tout les DRH de la conférence (je sais pas pour Alain) est le fait qu'ils prennent notes des remarques de leurs employés et veulent les garder motivés. On a donc une gestion humaine du personnel qui caractériserait le DRH pour le futur. On prend soin des employés. On essaye de trouver des réponses aux problèmes que rencontrent la société avec les employés qui sont eux aussi concernés tout compte fait. 
	
	-Un autre est le fait qu'ils se sont tous défini comme des agents de changement. Le DRH du futur est là pour faire bouger les choses, transformer la société.  Il est là pour apporter du changement. 
	