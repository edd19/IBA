Ce 4 mars 2015, notre groupe a eu l'occasion de participer à une table ronde accueillant trois directeurs de ressources humaines : Christine Thiran, Frédéric Nolf et Alain Goergen. Ce travail a pour objectif d'analyser leurs situations actuelles, de les mettre en relation avec la théorie dans le cadre du cours de Management Humain et de proposer un regard critique sur les discours et les limites de la théorie par rapport à l'évolution des pratiques quotidiennes des RH. \newline
\todo[inline]{mettre ça comme une section introduction}

La première est Christine Thiran qui travaille actuellement comme directrice des ressources humaines pour le groupe Mestdagh. Durant son parcours, elle a occupé le poste de DRH chez Belgacom durant dix années où elle a notamment été responsable d'un plan de restructuration concernant 4500 personnes. (S. VDD, 2002, \cite{planBEST})
Elle a ensuite travaillé pendant 11 ans aux cliniques universitaires de Saint-Luc où elle a mis en place une section ressources humaines.
Depuis cinq mois, elle travaille au sein du groupe Mestdagh, une entreprise familiale du secteur de la grande distribution. Ce secteur en grande difficulté disposant de marges limitées emploie majoritairement du personnel non-qualifié. Le groupe Mestdagh a comme vision d'offrir de l'emploi dans des zones économiquement défavorisées, et doit actuellement intégrer 16 anciennes enseignes (et leurs employés) du groupe Carrefour.
Durant cette courte période, Mme Thiran a émis prématurément un rapport d'étonnement questionnant entre autres les habitudes pour accélérer la prise de conscience des éléments à modifier. \newline

Ce faisant, Mme Thiran s'est identifiée à plusieurs rôles dont la gestion d'intégration des employés des anciennes enseignes Carrefour rachetées. Elle remarqua un clivage persistant entre les employés Carrefour et Mestdagh. Elle doit donc avoir un rôle de \textbf{fédérateur} et rassembler l'ensemble des employés Mestdagh autour d'une vision commune, d'une culture commune. \todo[inline]{Pe a completer} garder l'esprit familial de l'entreprise et sa proximité avec les gens, offrir des perspectives de carrière au sein des différents magasins. Concernant les recrutements, Mestdagh recrute principalement par interim.
\begin{center}
\textit{\og{}Dans la grande distribution chez nous [Mestdagh] on engage beaucoup par intérim.\fg{} (Nathalie Delobbe, 2015, \cite{tableronde})}	
\end{center}
Mme Thiran s'occupe principalement d'évaluer régulièrement les compétences de ses employés. Elle veillera à ce que les nouveaux employés aient une bonne capacité d'intégration et qu'ils savent garder le moral. \todo[inline]{à modifier si pas assez bien écrit?)} %\newline

Parallèlement, Mme Thiran se considère comme un \textbf{moteur de changement}. \todo[inline]{Il me semble que ça a été dit 2 paragraphes plus haut : ---> (AYM) non je ne pense pas je pense qu'on peut le laisser!)} Mestdagh évolue dans un secteur très compétitif où les marges de bénéfices sont très petites. De ce fait, cela a un impact sur l'organisation et donc sur la gestion des ressources humaines. Ainsi, son rôle est de secouer l'ensemble de ses collaborateurs. Via ce rapport d'étonnement, elle tente d'initier un changement, de faire évoluer la situation chez Mestdagh. \newline
Pour ce faire, une de ses pratiques est de pousser l'ensemble des employés mais aussi des syndicats à participer dans les prises de décisions. Pour ce faire, un dialogue constant entre la direction et les syndicats et employés et nécessaire. De plus, elle se voit comme le bras droit du CEO et veut entrer en symbiose avec celui-ci dans les prises de décisions. Elle se trouve ainsi dans une position optimale de par sa proximité avec à la fois le CEO, les employés et les syndicats.\newline

Le second intervenant était Frédéric Nolf, responsable des ressources humaines pour IBA depuis 2007. IBA est une société cotée en bourse proposant des équipements médicaux comme par exemple des salles de proton thérapie. Lors de sa création c'était une spin-off de l'UCL et elle est actuellement encore en alliance avec les universités pour trouver des nouveaux talents. 
% UTILE?? Son CEO a changé il y a trois ans.

Son rôle est d'assurer à l'entreprise qu'elle dispose des bonnes personnes pour son développement et que ces personnes soient dans la bonne entreprise pour leur propre développement personnel. Ce rôle s'articule autour de plusieurs missions : s'assurer de la motivation de son personnel, assurer un \og{} sens du focus\fg{}(nous y reviendrons), créer des liens de compétence à travers les divers départements d'IBA. \newline

Dans le cadre de ces missions, il doit relever de nombreux défis : faire rêver les ingénieurs nouvellement diplômés de travailler chez IBA, s'assurer que ces ingénieurs s'intègrent à la culture d'IBA, s'assurer que les ingénieurs travaillent réellement sur ce qui leur est demandé (\textit{\og{} quand il y a beaucoup de gens très intelligents, chacun a son idée sur ce qu'il faut faire \fg{}} (Nathalie Delobbe, 2015, \cite{tableronde}) ), informer les membres du personnel à travers le monde sur les projets actuels pour trouver les membres les plus aptes à les résoudre. \newline

Ces défis nécessitent de mettre en place certaines pratiques de GRH : faire des enquêtes de satisfaction et y répondre de manière concrète, mettre en place des activités communes (sport), prendre en compte toutes les idées proposées, créer des projets qui lient les convictions personnelles du personnel à la culture de la société (mise en place d'une cellule de réflexion pour le développement durable). \newline