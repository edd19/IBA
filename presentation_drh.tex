\section{Présentation des différents conférenciers}
La première intervenante est Christine Thiran qui travaille actuellement comme directrice des ressources humaines pour le groupe Mestdagh. Durant son parcours, elle a occupé le poste de DRH chez Belgacom durant dix années où elle a notamment été responsable d'un plan de restructuration concernant 4500 personnes (VDD., 2002).
Elle a ensuite travaillé pendant 11 ans aux Cliniques universitaires Saint-Luc où elle a mis en place une section ressources humaines.
Depuis cinq mois, elle travaille au sein du groupe Mestdagh, une entreprise familiale du secteur de la grande distribution. Ce secteur en grande difficulté, et disposant de marges limitées, emploie majoritairement du personnel non-qualifié. Le groupe Mestdagh a comme vision d'offrir de l'emploi dans des zones économiquement défavorisées, et doit actuellement intégrer 16 anciennes enseignes (et leurs employés) du groupe Carrefour.
Durant cette courte période, Mme Thiran a émis prématurément un rapport d'étonnement questionnant, entre autres, les habitudes, pour accélérer la prise de conscience des éléments à modifier. \newline

Ce faisant, Mme Thiran s'est identifiée à plusieurs rôles, dont la gestion d'intégration des employés des anciennes enseignes Carrefour rachetées. Elle remarqua un clivage persistant entre les employés Carrefour et Mestdagh. Elle doit donc avoir un rôle de fédérateur et rassembler l'ensemble des employés Mestdagh autour d'une vision et d'une culture commune, tout en gardant l'esprit familial de l'entreprise et sa proximité avec les gens.\newline

Parallèlement, Mme Thiran se considère comme un moteur de changement. Le contexte économique dans lequel évolue Mestdagh, a un impact sur l'organisation et donc sur la gestion des ressources humaines de l'entreprise. Ainsi, son rôle est de \og{}secouer\fg{} l'ensemble de ses collaborateurs. Ceci s'est déjà produit après la remise son rapport d'étonnement dans lequel elle tente de mettre en avant les éléments problématiques au sein de la société.\newline

Pour ce faire, une de ses pratiques est de pousser l'ensemble des employés, mais aussi des syndicats, à participer aux prises de décisions. Un dialogue constant entre la direction, les syndicats et employés est donc nécessaire. De plus, elle se voit comme le bras droit du CEO et veut être en symbiose avec celui-ci dans les prises de décisions. Elle se trouve ainsi dans une position optimale de par sa proximité à la fois avec le CEO, les employés et les syndicats.\newline

Le second intervenant est Frédéric Nolf, responsable des ressources humaines pour IBA depuis 2007. IBA est une société cotée en bourse, proposant des équipements médicaux comme par exemple des salles de protonthérapie. Lors de sa création, c'était une spin-off de l'UCL et elle est actuellement encore en partenariat avec des universités, pour trouver des nouveaux talents. \newline

Son rôle est d'assurer à l'entreprise qu'elle dispose des bonnes personnes pour son développement et que ces personnes soient dans la bonne entreprise pour leur développement personnel. Ce rôle s'articule autour de plusieurs missions : s'assurer de la motivation de son personnel, assurer un \og{}sens du focus\fg{}(ce sujet sera abordé plus en détail par la suite), créer des liens de compétence à travers les divers départements d'IBA. \newline

Dans le cadre de ces missions, il doit relever de nombreux défis : faire rêver les ingénieurs nouvellement diplômés de travailler chez IBA, s'assurer que ces ingénieurs s'intègrent à la culture d'IBA et qu'ils travaillent réellement sur ce qui leur est demandé (\textit{\og{} quand il y a beaucoup de gens très intelligents, chacun a son idée sur ce qu'il faut faire \fg{}} (Delobbe, \textit{et al}., 2015), informer les membres du personnel à travers le monde sur les projets actuels pour trouver les membres les plus aptes à y contribuer. \newline

Ces défis nécessitent de mettre en place certaines pratiques de DRH : faire des enquêtes de satisfaction et y répondre de manière concrète, mettre en place des activités communes (sport), prendre en compte toutes les idées proposées, créer des projets qui lient les convictions personnelles du personnel à la culture de la société (mise en place d'une cellule de réflexion pour le développement durable). \newline

Le dernier intervenant est Alain Goergen, directeur des ressources humaines de la police fédérale. Sa carrière s'est construite au sein de la police. Il commence par suivre une formation d'officier de gendarmerie à l'école militaire avec une licence en criminologie. Après 2 ans sur le terrain il entame une seconde formation en management pour se réorienter vers un poste administratif. Il a été en charge de nombreux projets toujours au sein des ressources humaines comme le recrutement, la formation ou encore la restructuration complète du département RH.\newline

La police fédérale est le plus grand employeur de Belgique. Elle compte 193 sections qui emploient plus de 50 000 personnes réparties dans différents domaines d'activités sur tout le territoire national. L'éventail de missions de la police est assez large: maintien de l'ordre, assistance aux victimes, missions anti-terroristes, etc.\\
Malgré la division entre les 192 polices locales et la police fédérale, elles sont toutes dépendantes du département RH de la police fédérale.\newline
 
La police est aujourd'hui dans une phase difficile, du point de vue économique et politique. En effet, le gouvernement doit faire des économies et ce service public n'est pas épargné. Le budget de la police est donc moindre que les années précédentes %[\textbf{trouver chiffres?} http://www.sudinfo.be/1166936/article/2014-12-10/impact-de-la-baisse-de-budget-pres-de-4000-policiers-en-moins-d-apres-vanessa-ma] PAS SPECIALEMENT UTILE APRES TOUT
 pour des résultats demandés identiques, voire même supérieurs. Nous pouvons illustrer ceci par les paroles d'Alain Goergen : \textit{\og Il y a de grosses difficultés budgétaires [\ldots] ce qui implique une optimalisation. Faire aussi bien voire mieux mais avec beaucoup moins de moyens \fg{}} (Delobbe, \textit{et al}., 2015). \newline

De plus, du fait des récents événements terroristes en France, l'armée Belge travaille en coopération avec la police pour assurer la sécurité des citoyens. Le fait d'être passé au niveau d'alerte 3 augmente le coût opérationnel des forces de l'ordre. Ceci demande donc de la coordination et des budgets plus conséquents. \newline

L'année passée, M. Goergen a eu la tâche de réorganiser le département RH. En effet celui-ci n'était pas un département en tant que tel puisqu'il s'agissait d'une des compétences de la direction générale, au même titre que la logistique ou l'ICT. Il a fallu diviser ces différents départements en entités individuelles.\newline

En plus de cela, une loi a été votée incitant les services publiques et donc la police à faire un pas vers la décentralisation. Ils ont donc choisi de créer 13 entités, sorte de sous-départements responsables d'une zone géographique. La décentralisation d'une compétence est faite si elle apporte une certaine valeur ajoutée. Une partie de celles-ci reste donc centralisée. Alain Goergen a donc mené cette restructuration interne de son département. Il doit maintenant modifier la structure des autres départements de la police.\newline

Finalement, il s'est défini comme un architecte qui a dû s'assurer du bon fonctionnement des changements, ou encore comme un équilibriste en s'assurant aussi que les changements s'effectuent bien pour les personnes concernées.\newline
