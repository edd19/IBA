%Comparaison avec modele ulrich
\section{Rôles des DRH dans le modèle d'Ulrich}

Nous avons à présent une vision des différents rôles que les intervenants endossent. Comme nous le savons, le modèle d'Ulrich donne quatre rôles que peut avoir un responsable de ressources humaines. Il peut donc être intéressant de replacer dans ce modèle, les différents DRH que nous avons rencontré. \\

Commençons par Alain Goergen. Il nous a décrit sa mission comme étant la restructuration complète du département des ressources humaines, et par la suite d'autres départements de la Police Fédérale. Mais il doit aussi aider les employés à s'habituer à ce changement. Au vue de ce rôle qu'il doit jouer, on voit très clairement que cela correspond à l'\og agent de changement \fg{} dans le modèle d'Ulrich. Pour rappel, ce dernier a pour but de faciliter le changement, donc faire en sorte que tout le monde accepte et s'adapte au changement.\\

Nous pouvons aussi le voir comme \og partenaire stratégique \fg{}. En effet, la décentralisation qu'il doit opérer, a un impacte stratégique en plus d'un impact sur les employés comme présenté ci-dessus. En reprenant les mots de Mr Goergen \og Le modus était tout ce qui peut être déconcentré doit être déconcentré mais si il n’y a pas de plus value à déconcentrer il faut le garder au niveau central \fg{} \footnote{TO DO : url video + minute}. Il parle ici de \og plus value \fg{}, il y a donc fait une analyse stratégique sur les compétences à décentraliser. Dans ce cas, il ne lui a pas été demandé d'appliquer un plan pensé en amont, mais de le créer par lui même. Ses décisions ont donc été prises en fonction de l'objectif d'efficacité à atteindre. Il est aussi important de se rappeler que la Police fait fasse à des réductions de budget et que ses décisions doivent les prendre en compte. Cela fait donc de lui un \og partenaire stratégique \fg{} au sens d'Ulrich.   


%Conclusion du la comparaison au modèle 