%Comparaison avec modele ulrich
\section{Rôles des DRH dans le modèle d'Ulrich}

Nous avons à présent une vision des différents rôles que les intervenants endossent. Comme nous le savons, le modèle d'Ulrich donne quatre rôles que peut avoir un responsable de ressources humaines. Il peut donc être intéressant de replacer dans ce modèle les différents DRH que nous avons rencontrés. \\

Commençons par Alain Goergen, celui-ci nous a décris sa mission comme étant la restructuration complète du département des ressources humaines, et par la suite d'autres départements de la police fédérale. Mais il doit aussi aider les employés à s'habituer à ce changement. Au vu de ce rôle qu'il doit jouer, on voit très clairement que cela correspond à l'\og Agent de changement \fg{} dans le modèle d'Ulrich. Pour rappel, ce dernier a pour but de faciliter le changement, donc faire en sorte que tout le monde accepte et s'adapte au changement.\\

Nous pouvons aussi le voir comme \og Partenaire stratégique \fg{}. En effet, la décentralisation qu'il doit opérer, a un impact stratégique en plus d'un impact sur les employés comme présenté ci-dessus. En reprenant les mots de M. Goergen \textit{\og Le modus était tout ce qui peut être déconcentré doit être déconcentré mais si il n'y a pas de plus value à déconcentrer il faut le garder au niveau central \fg{}} (Delobbe, \textit{et al}., 2015) . Il parle ici de \og plus-value \fg{}, il y a donc fait une analyse stratégique sur les compétences à décentraliser. Dans ce cas, il ne lui a pas été demandé d'appliquer un plan pensé en amont, mais de le créer par lui même. Ses décisions ont donc été prises en fonction de l'objectif d'efficacité à atteindre. Il est aussi important de se rappeler que la police fait face à des réductions de budget et que ses décisions doivent les prendre en compte. Cela fait donc de lui un \og Partenaire stratégique \fg{} au sens d'Ulrich.\\   

Frédéric Nolf, DRH d'IBA, peut être défini lui aussi comme \og{}Partenaire stratégique\fg{} dans le modèle d'Ulrich. Ce qui nous permet de le définir ainsi, est le rôle qu'il a tenu lors de la constitution de la \og dream team\fg{} pour le nouveau CEO. Lors de la création d'une telle équipe, il faut choisir des gens complémentaires, et qui gèreront l'entreprise pour qu'elle puisse atteindre ses objectifs. Ce rôle correspond bien à la définition du \og Partenaire stratégique \fg{} (Delobbe., 2015).\newline

Mais M. Nolf correspond aussi à un \og Champion des employés \fg{} dans le modèle d'Ulrich, dans la façon dont il traite le personnel d'IBA. En effet, il fait en sorte que les employés restent motivés par le travail qui les occupe chez IBA. La création d'activités communes comme le sport, ou de cellules de réflexion sur le développement durable permettent aux employés de s'épanouir au sein de la société. On a donc une conception très humaine de l'employé. On veut que chaque ingénieur et chercheur se sente bien dans la société. De plus, il s'assure que chaque personne fasse bien ce qui est attendue d'elle. \newline

Enfin, on peut considérer le DRH d'IBA comme un \og{}Agent de changement\fg{}. En effet, une autre de ses préoccupations est d'assurer l'adaptation des employés nécessaire suite à la modification de l'environnement. L'internationalisation d'une société pousse à des changements dans la culture d'une entreprise, et peut donc bouleverser l'environnement de la société. En effet, elle doit s'assouplir pour respecter les us et coutumes des pays dans lesquels elle s'installe. Chaque individu doit pouvoir s'attribuer et contribuer à la culture IBA. L'\og{}Agent de changement\fg{} doit accompagner ce changement pour qu'il se déroule le mieux possible. \newline

Terminons notre analyse par Mme Thiran. Comme M. Goergen et M. Nolf, elle correspond aussi à un \og Agent de changement \fg{} . Se décrivant elle-même comme un acteur de changement, elle doit ainsi s'occuper de l'intégration des employés de Carrefour à Mestdagh. Il lui faut donc gérer deux groupes de personnes ayant une culture différente pour qu'ils adoptent une vision commune au sein de l'entreprise Mestdagh. Mme Thiran se charge aussi de faire des rapports d'étonnement pour \og secouer \fg{} les différents acteurs de l'organisation, lorsqu'elle observe des choses qui doivent impérativement évoluer.\newline

Cependant, nous pourrions aussi positionner son rôle à un autre endroit sur le modèle d'Ulrich. En effet, elle s'apparente également au rôle de \og Championne des employés \fg{}. Considérant elle-même que le rôle d'un RH est de garder les gens motivés et de bonne humeur, Mme Thiran se charge de créer une bonne atmosphère de travail. Cela signifie donc pour elle : une bonne entente, de bonnes relations entre collègues et un esprit familial dans entreprise. 

%Conclusion du la comparaison au modèle 
Certains éléments du discours des intervenants ne peuvent pas être parfaitement intégrés dans le modèle d'Ulrich. Ceci nous conduit donc à la section suivante dans laquelle nous en ferons une analyse critique.
