%Comparaison avec modele ulrich
\section{Rôles des DRH dans le modèle d'Ulrich}

Nous avons à présent une vision des différents rôles que les intervenants endossent. Comme nous le savons, le modèle d'Ulrich donne quatre rôles que peut avoir un responsable de ressources humaines. Il peut donc être intéressant de replacer dans ce modèle, les différents DRH que nous avons rencontré. \\

Commençons par Alain Goergen. Il nous a décrit sa mission comme étant la restructuration complète du département des ressources humaines, et par la suite d'autres départements de la Police Fédérale. Mais il doit aussi aider les employés à s'habituer à ce changement. Au vue de ce rôle qu'il doit jouer, on voit très clairement que cela correspond à l'\og agent de changement \fg{} dans le modèle d'Ulrich. Pour rappel, ce dernier a pour but de faciliter le changement, donc faire en sorte que tout le monde accepte et s'adapte au changement. Mais il peut aussi en être l'initiateur, ce qui est le cas ici.

En lien avec le modèle d'Ulrich, on peut commencer par définir Alain Georgen comme un partenaire stratégique. Un partenaire stratégique doit assurer que le de déploiement des réformes mises en place permettent d'atteindre les objectifs recherchés. Cette restructuration a beaucoup de retombées dans différents domaines. A cause de la baisse de budget accordé à la Police, il a fallu redéfinir avec beaucoup d'attention les compétences à décentraliser ou non, prévoir les répercussions financières de celles-ci etc. Ceci a donc nécessité un travail en amont important.\\

Dans la continuité de cette restructuration interne, on peut définir Alain Goergen comme un agent changement. Un agent du changement est quelqu'un qui s'assure que l'évolution de l'entreprise dans laquelle il travaille se passe correctement, que les comportements des individus soient bien adaptés à l'environnement. C'est exactement ce rôle qu'a joue M. Georgen dans le cadre de cette restructuration. 

%Je pense que l'on pourrait aussi dire qu'il est partenaire startégique mais je ne sais pas s'il gere vraiment quelque chose en rapport avec la baisse de budget. C'est difficile pour la police en generale mais on ne sait pas ce qu'il doit faire en tant que DRH. 

%--> (du rapport police.tex)
%En lien avec le modèle d'Ulrich, on peut commencer par définir Alain Georgen comme un partenaire stratégique. Un partenaire stratégique doit assurer que le de déploiement des réformes mises en place permettent d'atteindre les objectifs recherchés. Cette restructuration a beaucoup de retombées dans différents domaines. A cause de la baisse de budget accordé à la Police, il a fallu redéfinir avec beaucoup d'attention les compétences à décentraliser ou non, prévoir les répercussions financières de celles-ci etc. Ceci a donc nécessité un travail en amont important.\\

%Conclusion du la comparaison au modèle 