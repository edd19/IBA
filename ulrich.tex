%Comparaison avec modele ulrich
\section{Rôles des DRH dans le modèle d'Ulrich}

Nous avons à présent une vision des différents rôles que les intervenants endossent. Comme nous le savons, le modèle d'Ulrich donne quatre rôles que peut avoir un responsable de ressources humaines. Il peut donc être intéressant de replacer dans ce modèle, les différents DRH que nous avons rencontré. \\

Commençons par Alain Goergen. Il nous a décrit sa mission comme étant la restructuration complète du département des ressources humaines, et par la suite d'autres départements de la Police Fédérale. Mais il doit aussi aider les employés à s'habituer à ce changement. Au vue de ce rôle qu'il doit jouer, on voit très clairement que cela correspond à l'\og agent de changement \fg{} dans le modèle d'Ulrich. Pour rappel, ce dernier a pour but de faciliter le changement, donc faire en sorte que tout le monde accepte et s'adapte au changement.\\

Nous pouvons aussi le voir comme \og partenaire stratégique \fg{}. En effet, la décentralisation qu'il doit opérer, a un impacte stratégique en plus d'un impact sur les employés comme présenté ci-dessus. En reprenant les mots de Mr Goergen \textit{\og Le modus était tout ce qui peut être déconcentré doit être déconcentré mais si il n’y a pas de plus value à déconcentrer il faut le garder au niveau central \fg{}} (Nathalie Delobbe, 2015, \cite{tableronde}) . Il parle ici de \og plus-value \fg{}, il y a donc fait une analyse stratégique sur les compétences à décentraliser. Dans ce cas, il ne lui a pas été demandé d'appliquer un plan pensé en amont, mais de le créer par lui même. Ses décisions ont donc été prises en fonction de l'objectif d'efficacité à atteindre. Il est aussi important de se rappeler que la Police fait fasse à des réductions de budget et que ses décisions doivent les prendre en compte. Cela fait donc de lui un \og partenaire stratégique \fg{} au sens d'Ulrich.\\   

Pour ce qui est de Frédéric Nolf, DRH d'IBA, on peut le classer lui aussi en tant que \og{}partenaire stratégique\fg{} dans le modèle d'Ulrich. Le rôle qu'il a joué durant la constitution d'une \og{}dream team\fg{} pour le nouveau CEO permet de le classer assez facilement dans cette catégorie. Mais ce qui conforte ceci est la façon dont Nolf profile les DRH de nos jours: \og{} \textit{Je pense que les DRH modèles [\ldots] se profilent tous comme des architectes et des transformateurs d’entreprise \ldots} \fg{} (Nathalie Delobbe, 2015, \cite{tableronde}). Il s'assure également que chaque chercheur puisse exploiter son potentiel tout en restant focalisant sur la tâche qui lui est assignée, les chercheurs étant selon lui des personnes qui ont souvent tendance à faire les choses à leur manière. \newline

Mais M. Nolf correspond aussi à un \og champion des employés \fg{} dans le modèle d'Ulrich dans la façon dont il traite le personnel d'IBA. En effet, il prend en compte les avis des gens afin que ces derniers ne se sentent pas lésés et restent motivés. De plus, la création d'activités communes comme le sport ou bien de cellules de réflexion sur le développement durable permettent aux employés de s'épanouir au sein de la société. On a donc une conception très humaine de l'employé. On veut que chaque ingénieur et chercheur se sente bien dans la société. \newline

Enfin, on peut considérer le GRH d'IBA comme un \og{} agent de changement \fg{}. En effet, une autre de ses préoccupations est d'exploiter le potentiel de chaque employé malgré l'internationalisation. L'internationalisation d'une société pousse à des changements dans la culture d'une entreprise. En effet, elle doit s'assouplir pour respecter les us et coutumes des pays dans lesquels la société s'installe. Chaque individu doit pouvoir s'attribuer et contribuer à la culture IBA.

Terminons notre analyse par Mme Thiran. Comme M. Goergen et M. Nolf, elle correspond aussi à un agent de changement. Se décrivant elle-même comme un acteur de changement, elle doit ainsi s’occuper de l’intégration des employés de Carrefour à Mestdagh, en gérant deux groupes de personnes ayant une culture différente pour qu’ils aient alors une vision commune au sein de l'entreprise Mestdagh. Mme Thiran se charge aussi de faire des rapports d’étonnement pour secouer les différents acteurs de l’organisation lorsqu’elle observe des choses qui ne vont pas pour ainsi faire bouger et évoluer les choses.\newline

Cependant, nous pourrions aussi positionner son rôle à un autre endroit sur le modèle d’Ulrich. En effet, elle s’apparente également au rôle de championne des employés. Considérant elle-même que le rôle d’un RH est de garder les gens motivés et de bonne humeur, Mme Thiran se charge de créer une atmosphère de travail telle qu’il y ait une bonne entente, de bonnes relations entre collègues et n’hésite pas à inciter les employés à exprimer leurs idées, pour ainsi créer un esprit familial dans l’entreprise. 

%Conclusion du la comparaison au modèle 
