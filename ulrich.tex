%Comparaison avec modele ulrich
\section{Rôles des DRH dans le modèle d'Ulrich}

Nous avons à présent une vision des différents rôles que les intervenants endossent. Comme nous le savons, le modèle d'Ulrich donne quatre rôles que peut avoir un responsable de ressources humaines. Il peut donc être intéressant de replacer dans ce modèle, les différents DRH que nous avons rencontré. \\

Commençons par Alain Goergen. Il nous a décrit sa mission comme étant la restructuration complète du département des ressources humaines, et par la suite d'autres départements de la Police Fédérale. Mais il doit aussi aider les employés à s'habituer à ce changement. Au vue de ce rôle qu'il doit jouer, on voit très clairement que cela correspond à l'\og agent de changement \fg{} dans le modèle d'Ulrich. Pour rappel, ce dernier a pour but de faciliter le changement, donc faire en sorte que tout le monde accepte et s'adapte au changement.\\

%En lien avec le modèle d'Ulrich, on peut commencer par définir Alain Georgen comme un partenaire stratégique. Un partenaire stratégique doit assurer que le de déploiement des réformes mises en place permettent d'atteindre les objectifs recherchés. Cette restructuration a beaucoup de retombées dans différents domaines. A cause de la baisse de budget accordé à la Police, il a fallu redéfinir avec beaucoup d'attention les compétences à décentraliser ou non, prévoir les répercussions financières de celles-ci etc. Ceci a donc nécessité un travail en amont important.\\

%Dans la continuité de cette restructuration interne, on peut définir Alain Goergen comme un agent changement. Un agent du changement est quelqu'un qui s'assure que l'évolution de l'entreprise dans laquelle il travaille se passe correctement, que les comportements des individus soient bien adaptés à l'environnement. C'est exactement ce rôle qu'a joue M. Georgen dans le cadre de cette restructuration. 

Nous pouvons aussi le voir comme \og partenaire stratégique \fg{}. En effet, la décentralisation qu'il doit opérer, a un impacte stratégique en plus d'un impact sur les employés comme présenté ci-dessus. En reprenant les mots de Mr Goergen \og Le modus était tout ce qui peut être déconcentré doit être déconcentré mais si il n’y a pas de plus value à déconcentrer il faut le garder au niveau central \fg{} \footnote{TO DO : url video + minute}. Il parle ici de \og plus value \fg{}, il y a donc fait une analyse stratégique sur les compétences à décentraliser. Dans ce cas, il ne lui a pas été demandé d'appliquer un plan pensé en amont, mais de le créer par lui même. Ses décisions ont donc été prises en fonction de l'objectif d'efficacité à atteindre. Il est aussi important de se rappeler que la Police fait fasse à des réductions de budget et que ses décisions doivent les prendre en compte. Cela fait donc de lui un \og partenaire stratégique \fg{} au sens d'Ulrich.\\   

\todo[inline]{This is a shaft}

Pour ce qui est de Frédéric Nolf, GRH d'IBA, on peut le classer lui aussi en tant que \textbf{agent de changement} dans le modèle d'Ulrich. Le rôle qu'il a joué durant la constitution d'une dream team pour le nouveau CEO permet de le classer assez facilement dans cette catégorie. Mais ce qui conforte ceci est la façon dont Nolf profile les DRH de nos jours:\\

<< \textit{je pense que les DRH modèles ou en tout cas, se profilent tous comme 
des architectes et des transformateurs d’entreprise ...}>> \\

Mais Nolf correspond aussi à un \textbf{champion des employés} dans le modéle d'Ulrich dans la façon dont il traite le personnel d'IBA. En effet, il prend compte des avis des gens afin que ces derniers ne se sentent pas laisés et restent motivés. De plus, la création d'activité communes comme le sport ou bien de cellules de réflexion sur le développement durable permettent aux employés de s'épanouir au sein de la société. On a donc une conception très humaine de l'employé. On veut qu'il , le chercheur, se sente bien dans la société.\\

Enfin, on peut considérer le GRH d'IBA comme un \textbf{partenaire stratégique}. Un des soucis de Frédéric Nolf, comme indiquer plus haut, est de garder les employés concentrés dans leurs tâches. Les chercheurs sont des gens qui font facilement des choses à leur manière. De plus, une autre de ses préoccupations est d'exploiter le potentiel de chaque employé malgré l'internationalisation. Tout ceci est dans l'optique d'augmenter la productivité, l'efficacité de l'entreprise. Produire plus mais avec le même effectif. 



%Conclusion du la comparaison au modèle 